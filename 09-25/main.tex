\documentclass[a4paper,11pt]{article}
\usepackage[a4paper, margin=8em]{geometry}

% usa i pacchetti per la scrittura in italiano
\usepackage[french,italian]{babel}
\usepackage[T1]{fontenc}
\usepackage[utf8]{inputenc}
\frenchspacing 

% usa i pacchetti per la formattazione matematica
\usepackage{amsmath, amssymb, amsthm, amsfonts}

% usa altri pacchetti
\usepackage{gensymb}
\usepackage{hyperref}
\usepackage{standalone}

% imposta il titolo
\title{Appunti Elettrotecnica}
\author{Luca Seggiani}
\date{25-09-24}

% imposta lo stile
% usa helvetica
\usepackage[scaled]{helvet}
% usa palatino
\usepackage{palatino}
% usa un font monospazio guardabile
\usepackage{lmodern}

\renewcommand{\rmdefault}{ppl}
\renewcommand{\sfdefault}{phv}
\renewcommand{\ttdefault}{lmtt}

% disponi teoremi
\usepackage{tcolorbox}
\newtcolorbox[auto counter, number within=section]{theorem}[2][]{%
	colback=blue!10, 
	colframe=blue!40!black, 
	sharp corners=northwest,
	fonttitle=\sffamily\bfseries, 
	title=Teorema~\thetcbcounter: #2, 
	#1
}


% disegni
\usepackage{pgfplots}
\pgfplotsset{width=10cm,compat=1.9}

% disponi definizioni
\newtcolorbox[auto counter, number within=section]{definition}[2][]{%
	colback=red!10,
	colframe=red!40!black,
	sharp corners=northwest,
	fonttitle=\sffamily\bfseries,
	title=Definizione~\thetcbcounter: #2,
	#1
}

% disponi codice
\usepackage{listings}
\usepackage[table]{xcolor}

\lstdefinestyle{codestyle}{
		backgroundcolor=\color{black!5}, 
		commentstyle=\color{codegreen},
		keywordstyle=\bfseries\color{magenta},
		numberstyle=\sffamily\tiny\color{black!60},
		stringstyle=\color{green!50!black},
		basicstyle=\ttfamily\footnotesize,
		breakatwhitespace=false,         
		breaklines=true,                 
		captionpos=b,                    
		keepspaces=true,                 
		numbers=left,                    
		numbersep=5pt,                  
		showspaces=false,                
		showstringspaces=false,
		showtabs=false,                  
		tabsize=2
}

\lstdefinestyle{shellstyle}{
		backgroundcolor=\color{black!5}, 
		basicstyle=\ttfamily\footnotesize\color{black}, 
		commentstyle=\color{black}, 
		keywordstyle=\color{black},
		numberstyle=\color{black!5},
		stringstyle=\color{black}, 
		showspaces=false,
		showstringspaces=false, 
		showtabs=false, 
		tabsize=2, 
		numbers=none, 
		breaklines=true
}

\lstdefinelanguage{javascript}{
	keywords={typeof, new, true, false, catch, function, return, null, catch, switch, var, if, in, while, do, else, case, break},
	keywordstyle=\color{blue}\bfseries,
	ndkeywords={class, export, boolean, throw, implements, import, this},
	ndkeywordstyle=\color{darkgray}\bfseries,
	identifierstyle=\color{black},
	sensitive=false,
	comment=[l]{//},
	morecomment=[s]{/*}{*/},
	commentstyle=\color{purple}\ttfamily,
	stringstyle=\color{red}\ttfamily,
	morestring=[b]',
	morestring=[b]"
}

% disponi sezioni
\usepackage{titlesec}

\titleformat{\section}
	{\sffamily\Large\bfseries} 
	{\thesection}{1em}{} 
\titleformat{\subsection}
	{\sffamily\large\bfseries}   
	{\thesubsection}{1em}{} 
\titleformat{\subsubsection}
	{\sffamily\normalsize\bfseries} 
	{\thesubsubsection}{1em}{}

% disponi alberi
\usepackage{forest}

\forestset{
	rectstyle/.style={
		for tree={rectangle,draw,font=\large\sffamily}
	},
	roundstyle/.style={
		for tree={circle,draw,font=\large}
	}
}

% disponi algoritmi
\usepackage{algorithm}
\usepackage{algorithmic}
\makeatletter
\renewcommand{\ALG@name}{Algoritmo}
\makeatother

% disponi numeri di pagina
\usepackage{fancyhdr}
\fancyhf{} 
\fancyfoot[L]{\sffamily{\thepage}}

\makeatletter
\fancyhead[L]{\raisebox{1ex}[0pt][0pt]{\sffamily{\@title \ \@date}}} 
\fancyhead[R]{\raisebox{1ex}[0pt][0pt]{\sffamily{\@author}}}
\makeatother

\begin{document}
% sezione (data)
\section{Lezione del 25-09-24}

% stili pagina
\thispagestyle{empty}
\pagestyle{fancy}

% testo
\subsection{Introduzione}
Il corso di elettrotecnica riguarda lo studio dei \textbf{circuiti elettrici} e dei \textbf{macchinari elettrici}.

\subsubsection{Analisi dei circuiti elettrici}

Le leggi di Maxwell vanno a descrivere come si evolvono, nel tempo e nello spazio, i campi elettrici e magnetici.
Purtroppo,  le equazioni di Maxwell sono equazioni differenziali e legate fra di loro, ergo si possono spesso avere solo soluzioni numeriche.
Esistono però casì particolari in cui si possono fare semplificazioni considerevoli.

Un \textbf{circuito elettrico} è formato da fili conduttori e \textbf{componenti circuitali}.
All'interno di un circuito si va a descrivere un'onda elettrica:

$$
	\psi(s,t)
$$

rappresentata come una funzione di spazio e tempo.
Poniamo ad esempio la funzione, sulla sola posizione x:

$$
	\psi(x, t) = y \sin{\left( \frac{2\pi}{\lambda}x - \frac{2\pi}{T} t \right)}
$$

Questa funzione ha comunque due variabili: la posizione $x$ e il tempo $t$.
Immaginiamo di prendere un punto $x_0$ sul circuito elettrico:

$$
	\psi(t) = y \sin{\left( \frac{2\pi}{\lambda}x_0 - \frac{2\pi}{T} t \right)}
$$

Con $x_0 = 0$, annulliamo il primo termine.
A questo punto abbiamo ottenuto una funzione in una sola variabile:

$$
\psi(x_0, t) = y \sin{\left( - \frac{2\pi}{T} t \right)}
$$

ovvero una sinusoide invertita che oscilla fra un massimo di $y$ e un minimo di $-y$.

Questo significa che, mettendoci sul punto $x_0 = 0$ del circuito elettrico, notiamo che il valore dell'onda elettrica varia nel tempo seguendo questa funzione sinousidale.

Possiamo fare il processo invrso: fissiamo il tempo $t$, e vediamo come varia l'onda elettrica su diverse posizioni $x$ nel circuito.
Abbiamo, simbolicamente:

$$
	\psi(x) = y \sin{\left( \frac{2\pi}{\lambda}x - \frac{2\pi}{T} t_0 \right)}
$$

da cui ricaviamo l'equazione in una sola variabile $t$:

$$
	\psi(x) = y \sin{\left( \frac{2\pi}{\lambda}x \right)}
$$

ovvero una sinusoide che, come prima, oscilla fra un massimo di $y$ e un minimo di $-y$.
Si riporta un grafico:

\begin{center}

\begin{tikzpicture}
    \begin{axis}[
        xlabel={$x$},
        ylabel={$\psi(x)$},
        domain=-10:10, % set the x range you want
				samples=100,
        grid=major, % add a grid
				ytick={-2, 2},
				yticklabels={$-y$, $y$},
				ymin = -3, ymax = 3,
				width=15cm,
				height=7cm
    ]
    \addplot[
        blue,
        thick
    ] {2 * sin(50*x)}; 
    \end{axis}
\end{tikzpicture}

\end{center}

Questo significa che, all'istante $t_0 = 0$ notiamo che il valore dell'onda elettrica varia sulla lunghezza del circuito seguendo ancora questa funzione sinousidale.

Possiamo provare a calcolare lunghezza d'onda e periodo di questa oscillazione: visto che il periodo del seno è $2\pi$, abbiamo che nello spazio la lunghezza d'onda è $\lambda$ e nel tempo il periodo è $T$.

Proviamo a calcolare $\lambda$: sappiamo che la lunghezza d'onda equivale alla velocità di propagazione sulla frequenza dell'oscillazione, ovvero:

$$
\lambda = \frac{v}{f}
$$

Posti i valori $300 \cdot 10^6 \ \mathrm{m/s}$ per $v$ e $50 \ \text{Hz}$ per $f$ (la frequenza della rete elettrica), abbiamo:

$$
\lambda = \frac{3.00 \cdot 10^6 \ \mathrm{m/s}}{50 \ \text{Hz}} = 6000 \ \text{km} 
$$

Questa lunghezza d'onda diventa rilevante in trasmissioni elettriche su larga scala.

Possiamo fare considerazioni diverse se prendiamo in esempio le comunicazioni radio: lì si parla di frequenze $f >> 50 \ \text{Hz}$, nell'ordine dei megahertz o gigahertz.

L'elevata velocità della corrente ci permette di fare un'importante approssimazione e considerare \textbf{circuiti a parametri concetrati}.
Quest'ipotesi, in inglese \textit{lumped element model}, ci permette di ignorare l'estensione fisica del circuito, e quindi le variazioni delle funzioni d'onda sulla variabille spazio $s$, concentradosi sulla variabile tempo $t$.

\subsection{Grandezze}
Si usano le seguenti grandezze:
\subsubsection{Intensità di corrente}

\begin{definition}{Corrente elettrica}	
Si indica con $I$ la corrente elettrica, misurata in Ampere ($\mathrm{A}$), e definita come la variazione di carica:
$$
I = \frac{dq}{dt}
$$
\end{definition}

Si prende come positivo il verso in cui si muovono i portatori di carica positive, anche se sappiamo nella stragrande maggioranza dei casi i portatori di carica essere negativi, e quindi il movimento vero e proprio degli elettroni in direzione opposta.

Notiamo che se un segmento di circuito da $A$ a $B$ si ha una corrente $I_{AB}$, vale:

$$
I_{AB} = -I_{BA}
$$

\subsubsection{Differenza di potenziale}

\begin{definition}{Differenza di potenziale}
Si indica con $V$ la differenza di potenziale o \textit{tensione}, misurata in Volt ($\mathrm{V}$), e definita come il lavoro necessario a spostare una carica elementare positiva da un punto $A$ ad un punto $B$ sulla carica:

$$
	V_{AB}(t) = \frac{L_{AB}(t)}{q(t)}
$$
\end{definition}

Il segno del potenziale è definito come \textit{positivo} quando si deve vincere il campo magnetico per spingere la carica, ergo il campo elettrico svolge lavoro \textit{negativo} sulla carica.
Come prima, su segmenti di circuito da $A$ a $B$ vale:

$$
V_{AB} = -V_{BA}
$$

\subsubsection{Nota sul dipolo elettrico}
I componenti circuitali, presi a sé, vengono detti \textbf{dipoli elettrici}, dal fatto che hanno 2 poli.
Di un dipolo elettrico si può misurare la differenza di potenziale ai capi e la corrente che vi scorre attraverso.

Quando si parla di tensione, o si parla di differenze di potenziale, o si assume un riferimento (lo zero del potenziale).
Non possiamo sapere a priori se il potenziale al capo di un dipolo è maggiore del potenziale all'altro capo: bisogna prima scegliere una direzione e poi vedere se il segno ricavato è concorde o meno con la nostra scelta.

Lo stesso vale per la corrente.
I riferimenti concordi al verso della corrente si dicono \textbf{associati}, quelli discordi si dicono \textbf{non associati}.

\subsubsection{Potenza elettrica}

\begin{definition}{Potenza elettrica}
Si indica con $P$ la potenza elettrica, misurata in Watt ($\mathrm{W}$) e definita come il prodotto:
$$
	P = IV
$$
fra corrente e tensione.
\end{definition}

Anche la potenza ha un segno, che in questo caso si riferisce a potenza \textit{erogata} o \textit{dissipata}.
La potenza calcolata sui riferimenti associati positiva è dissipata, quella negativa è erogata.
Viceversa, la potenza calcolata sui riferimenti non associati positiva è erogata, quella negativa è dissipata.

\subsubsection{Energia}

\begin{definition}{Energia}
Si indica con $W$ (non Watt!) l'energia, misurata in Joule ($\mathrm{J}$), o in Kilowatt/ora ($\mathrm{KW/h}$), e definita come l'integrale sul tempo della potenza:
$$
W = \int_{-\infty}^t P \ dt 
$$
\end{definition}

\subsection{Leggi di Kirchoff}
Iniziamo col dare dei nomi a particolari punti del circuito elettrico: i punti di incontro di più fili prendono il nome di \textbf{nodi}, e i fili che collegano i dipoli ai nodi prendono il nome di \textbf{rami}.
Da questo abbiamo che nei nodi si incontrano 3 o più rami.

Da qui possiamo definire la legge:
\begin{theorem}{Prima legge di Kirchoff}
		La somma algebrica delle correnti dei rami tagliati da una linea chiusa è uguale a 0.
		In particolare, la somma algebrica delle correnti entranti e uscenti da un nodo è uguale a 0.
\end{theorem}

Definiamo quindi il concetto di \textbf{maglia}: una maglia è un percorso chiuso di nodi e rami, ovvero un sottoinsieme di rami tali per cui spostandosi da un nodo all'altro si percorre ogni ramo una sola volta.
Sulle maglie si ha:

\begin{theorem}{Seconda legge di Kirchoff}
	La somma algebrica delle cadute di potenziale lungo una maglia è uguale a 0.
\end{theorem}

\end{document}
