
\documentclass[a4paper,11pt]{article}
\usepackage[a4paper, margin=8em]{geometry}

% usa i pacchetti per la scrittura in italiano
\usepackage[french,italian]{babel}
\usepackage[T1]{fontenc}
\usepackage[utf8]{inputenc}
\frenchspacing 

% usa i pacchetti per la formattazione matematica
\usepackage{amsmath, amssymb, amsthm, amsfonts}

% usa altri pacchetti
\usepackage{gensymb}
\usepackage{hyperref}
\usepackage{standalone}

% imposta il titolo
\title{Appunti Elettrotecnica}
\author{Luca Seggiani}
\date{2024}

% imposta lo stile
% usa helvetica
\usepackage[scaled]{helvet}
% usa palatino
\usepackage{palatino}
% usa un font monospazio guardabile
\usepackage{lmodern}

\renewcommand{\rmdefault}{ppl}
\renewcommand{\sfdefault}{phv}
\renewcommand{\ttdefault}{lmtt}

% disponi il titolo
\makeatletter
\renewcommand{\maketitle} {
	\begin{center} 
		\begin{minipage}[t]{.8\textwidth}
			\textsf{\huge\bfseries \@title} 
		\end{minipage}%
		\begin{minipage}[t]{.2\textwidth}
			\raggedleft \vspace{-1.65em}
			\textsf{\small \@author} \vfill
			\textsf{\small \@date}
		\end{minipage}
		\par
	\end{center}

	\thispagestyle{empty}
	\pagestyle{fancy}
}
\makeatother

% disponi teoremi
\usepackage{tcolorbox}
\newtcolorbox[auto counter, number within=section]{theorem}[2][]{%
	colback=blue!10, 
	colframe=blue!40!black, 
	sharp corners=northwest,
	fonttitle=\sffamily\bfseries, 
	title=~\thetcbcounter: #2, 
	#1
}

% disponi definizioni
\newtcolorbox[auto counter, number within=section]{definition}[2][]{%
	colback=red!10,
	colframe=red!40!black,
	sharp corners=northwest,
	fonttitle=\sffamily\bfseries,
	title=~\thetcbcounter: #2,
	#1
}

% U.D.M
\newcommand{\amp}{\ensuremath{\mathrm{A}}}
\newcommand{\volt}{\ensuremath{\mathrm{V}}}
\newcommand{\meter}{\ensuremath{\mathrm{m}}}
\newcommand{\second}{\ensuremath{\mathrm{s}}}
\newcommand{\farad}{\ensuremath{\mathrm{F}}}
\newcommand{\henry}{\ensuremath{\mathrm{H}}}
\newcommand{\siemens}{\ensuremath{\mathrm{S}}}

% circuiti
\usepackage{circuitikz}
\usetikzlibrary{babel}

% disegni
\usepackage{pgfplots}
\pgfplotsset{width=10cm,compat=1.9}

% disponi codice
\usepackage{listings}
\usepackage[table]{xcolor}

\lstdefinestyle{codestyle}{
		backgroundcolor=\color{black!5}, 
		commentstyle=\color{codegreen},
		keywordstyle=\bfseries\color{magenta},
		numberstyle=\sffamily\tiny\color{black!60},
		stringstyle=\color{green!50!black},
		basicstyle=\ttfamily\footnotesize,
		breakatwhitespace=false,         
		breaklines=true,                 
		captionpos=b,                    
		keepspaces=true,                 
		numbers=left,                    
		numbersep=5pt,                  
		showspaces=false,                
		showstringspaces=false,
		showtabs=false,                  
		tabsize=2
}

\lstdefinestyle{shellstyle}{
		backgroundcolor=\color{black!5}, 
		basicstyle=\ttfamily\footnotesize\color{black}, 
		commentstyle=\color{black}, 
		keywordstyle=\color{black},
		numberstyle=\color{black!5},
		stringstyle=\color{black}, 
		showspaces=false,
		showstringspaces=false, 
		showtabs=false, 
		tabsize=2, 
		numbers=none, 
		breaklines=true
}

\lstdefinelanguage{javascript}{
	keywords={typeof, new, true, false, catch, function, return, null, catch, switch, var, if, in, while, do, else, case, break},
	keywordstyle=\color{blue}\bfseries,
	ndkeywords={class, export, boolean, throw, implements, import, this},
	ndkeywordstyle=\color{darkgray}\bfseries,
	identifierstyle=\color{black},
	sensitive=false,
	comment=[l]{//},
	morecomment=[s]{/*}{*/},
	commentstyle=\color{purple}\ttfamily,
	stringstyle=\color{red}\ttfamily,
	morestring=[b]',
	morestring=[b]"
}

% disponi sezioni
\usepackage{titlesec}

\titleformat{\section}
	{\sffamily\Large\bfseries} 
	{\thesection}{1em}{} 
\titleformat{\subsection}
	{\sffamily\large\bfseries}   
	{\thesubsection}{1em}{} 
\titleformat{\subsubsection}
	{\sffamily\normalsize\bfseries} 
	{\thesubsubsection}{1em}{}

% disponi alberi
\usepackage{forest}

\forestset{
	rectstyle/.style={
		for tree={rectangle,draw,font=\large\sffamily}
	},
	roundstyle/.style={
		for tree={circle,draw,font=\large}
	}
}

% disponi algoritmi
\usepackage{algorithm}
\usepackage{algorithmic}
\makeatletter
\renewcommand{\ALG@name}{Algoritmo}
\makeatother

% disponi numeri di pagina
\usepackage{fancyhdr}
\fancyhf{} 
\fancyfoot[L]{\sffamily{\thepage}}

\makeatletter
\fancyhead[L]{\raisebox{1ex}[0pt][0pt]{\sffamily{\@title \ \@date}}} 
\fancyhead[R]{\raisebox{1ex}[0pt][0pt]{\sffamily{\@author}}}
\makeatother

\begin{document}
% sezione (data)
\section{Lezione del 03-10-24}

% stili pagina
\thispagestyle{empty}
\pagestyle{fancy}

% testo
\subsection{Principio di sovrapposizione degli effetti}
Possiamo sfruttare la linearità delle equazioni che descrivono i circuiti visti finora per dimostrare il seguente risultato:
\begin{theorem}{Principio di sovrapposizione}
	La risposta di una rete alla sollecitazione di più generatori indipendenti può essere ottenuta considerando ciascun generatore \textit{separatamente attivo} e sommando algebricamente le risposte.
\end{theorem}

Nel dettaglio, in un circuito con generatori di voltaggio e tensione, questo significa prendere ogni generatore, uno per volta, e considerarlo come l'unico attivo, disattivando gli altri.
Disattivare un generatore significa impostare il voltaggio (o la corrente) emessa a 0. Quindi:
\begin{itemize}
	\item \textbf{Generatori di tensione:} disattivare significa trasformare in cortocircuiti ($e(t) = 0$);
	\item \textbf{Generatori di corrente:} disattivare significa trasformare in circuiti aperti ($I(t) = 0$).
\end{itemize}

Algebricamente, se vogliamo ricavare la corrente o il voltaggio su una maglia di un circuito con $n$ generatori, possiamo porla come la somma delle $I'_x + I''_x + ... + I_x^n$ correnti ($V'_x + V''_x + ... + V_x^n$ voltaggi) ricavati da ogni generatore preso singolarmente.

\par\smallskip

Facciamo un'esempio:

\begin{center}
\begin{circuitikz}
    \draw (2,-2) 
				-- (0,-2) 
				to[ european voltage source, v=$e(t)$] (0, 0)
				to[R, l=$R$] (0,2);	
		
		\draw (2,2 )to[ european current source, i=$J(t)$] (2,-2);
		\draw (4,2 )to[R, l=$R$] (4,-2);
		
    % Draw horizontal lines at the top and bottom
    \draw (0,2) -- (4,2);
    \draw (0,-2) -- (4,-2);

\end{circuitikz}
\end{center}

Poniamo di voler calcolare la potenza dissipata dal resistore destro.
Potremo applicare quanto ricavato da Ohm, ergo:
$$
P_R = R I_x^2
$$
con $I_x$ la corrente che passa sul resistore destro.

Questa corrente si calcola agilmente attraverso il principio di sovrapposizione.
Poniamo quindi $I_x = I_x' + I_x''$, con $I_x'$ data dal generatore di tensione $e(t)$, e $I_x''$ data dal generatore di corrente $J(t)$.

Calcoliamo quindi le correnti trovate in entrambi i casi:
\begin{itemize}
	\item $I_x'$, \textbf{generatore di tensione attivo.} 
		Si ha che il generatore di corrente, disattivato, equivale al circuito aperto, ergo l'unica maglia è quella che collega in serie le due resistenze.
		Da qui si calcola velocemente che:
		$$ I_x' = \frac{e(t)}{2R} $$
		oppure si applica la seconda legge di Kirchoff:
		$$ e(t) - I_x' R - I_x'R = 0 $$
		da cui si ricava lo stesso risultato.
	\item $I_x''$, \textbf{generatore di corrente attivo.}
		Si ha che il generatore di tensione, disattivato, equivale al circuito chiuso, ergo abbiamo quello che è effettivamente un partitore di corrente formato da due resistenze in parallelo.
		Visto che le due resistenze sono uguali, possimo dire che la corrente è partizionata ugualmente, facendo attenzione ai segni (il generatore di corrente ha direzione opposta al generatore di tensione), cioè:
		$$ I_x'' = -J(t) \frac{G_R}{G_{tot.}} = -J(t) \frac{R}{R + R} = -\frac{J(t)}{2}$$
\end{itemize}

A questo punto otteniamo la corrente totale come:
$$ I_x = I_x' + I_x'' = \frac{e(t)}{2R} - \frac{J(t)}{2} $$
e la potenza come:
$$ P_R = R \left( \frac{e(t)}{2R} - \frac{J(t)}{2} \right)^2 $$



\end{document}
