
\documentclass[a4paper,11pt]{article}
\usepackage[a4paper, margin=8em]{geometry}

% usa i pacchetti per la scrittura in italiano
\usepackage[french,italian]{babel}
\usepackage[T1]{fontenc}
\usepackage[utf8]{inputenc}
\frenchspacing 

% usa i pacchetti per la formattazione matematica
\usepackage{amsmath, amssymb, amsthm, amsfonts}

% usa altri pacchetti
\usepackage{gensymb}
\usepackage{hyperref}
\usepackage{standalone}

% imposta il titolo
\title{Appunti Elettrotecnica}
\author{Luca Seggiani}
\date{2024}

% imposta lo stile
% usa helvetica
\usepackage[scaled]{helvet}
% usa palatino
\usepackage{palatino}
% usa un font monospazio guardabile
\usepackage{lmodern}

\renewcommand{\rmdefault}{ppl}
\renewcommand{\sfdefault}{phv}
\renewcommand{\ttdefault}{lmtt}

% disponi il titolo
\makeatletter
\renewcommand{\maketitle} {
	\begin{center} 
		\begin{minipage}[t]{.8\textwidth}
			\textsf{\huge\bfseries \@title} 
		\end{minipage}%
		\begin{minipage}[t]{.2\textwidth}
			\raggedleft \vspace{-1.65em}
			\textsf{\small \@author} \vfill
			\textsf{\small \@date}
		\end{minipage}
		\par
	\end{center}

	\thispagestyle{empty}
	\pagestyle{fancy}
}
\makeatother

% disponi teoremi
\usepackage{tcolorbox}
\newtcolorbox[auto counter, number within=section]{theorem}[2][]{%
	colback=blue!10, 
	colframe=blue!40!black, 
	sharp corners=northwest,
	fonttitle=\sffamily\bfseries, 
	title=~\thetcbcounter: #2, 
	#1
}

% disponi definizioni
\newtcolorbox[auto counter, number within=section]{definition}[2][]{%
	colback=red!10,
	colframe=red!40!black,
	sharp corners=northwest,
	fonttitle=\sffamily\bfseries,
	title=~\thetcbcounter: #2,
	#1
}

% U.D.M
\newcommand{\amp}{\ensuremath{\mathrm{A}}}
\newcommand{\volt}{\ensuremath{\mathrm{V}}}
\newcommand{\meter}{\ensuremath{\mathrm{m}}}
\newcommand{\second}{\ensuremath{\mathrm{s}}}
\newcommand{\farad}{\ensuremath{\mathrm{F}}}
\newcommand{\henry}{\ensuremath{\mathrm{H}}}
\newcommand{\siemens}{\ensuremath{\mathrm{S}}}

% circuiti
\usepackage{circuitikz}
\usetikzlibrary{babel}

% disegni
\usepackage{pgfplots}
\pgfplotsset{width=10cm,compat=1.9}

% disponi codice
\usepackage{listings}
\usepackage[table]{xcolor}

\lstdefinestyle{codestyle}{
		backgroundcolor=\color{black!5}, 
		commentstyle=\color{codegreen},
		keywordstyle=\bfseries\color{magenta},
		numberstyle=\sffamily\tiny\color{black!60},
		stringstyle=\color{green!50!black},
		basicstyle=\ttfamily\footnotesize,
		breakatwhitespace=false,         
		breaklines=true,                 
		captionpos=b,                    
		keepspaces=true,                 
		numbers=left,                    
		numbersep=5pt,                  
		showspaces=false,                
		showstringspaces=false,
		showtabs=false,                  
		tabsize=2
}

\lstdefinestyle{shellstyle}{
		backgroundcolor=\color{black!5}, 
		basicstyle=\ttfamily\footnotesize\color{black}, 
		commentstyle=\color{black}, 
		keywordstyle=\color{black},
		numberstyle=\color{black!5},
		stringstyle=\color{black}, 
		showspaces=false,
		showstringspaces=false, 
		showtabs=false, 
		tabsize=2, 
		numbers=none, 
		breaklines=true
}

\lstdefinelanguage{javascript}{
	keywords={typeof, new, true, false, catch, function, return, null, catch, switch, var, if, in, while, do, else, case, break},
	keywordstyle=\color{blue}\bfseries,
	ndkeywords={class, export, boolean, throw, implements, import, this},
	ndkeywordstyle=\color{darkgray}\bfseries,
	identifierstyle=\color{black},
	sensitive=false,
	comment=[l]{//},
	morecomment=[s]{/*}{*/},
	commentstyle=\color{purple}\ttfamily,
	stringstyle=\color{red}\ttfamily,
	morestring=[b]',
	morestring=[b]"
}

% disponi sezioni
\usepackage{titlesec}

\titleformat{\section}
	{\sffamily\Large\bfseries} 
	{\thesection}{1em}{} 
\titleformat{\subsection}
	{\sffamily\large\bfseries}   
	{\thesubsection}{1em}{} 
\titleformat{\subsubsection}
	{\sffamily\normalsize\bfseries} 
	{\thesubsubsection}{1em}{}

% disponi alberi
\usepackage{forest}

\forestset{
	rectstyle/.style={
		for tree={rectangle,draw,font=\large\sffamily}
	},
	roundstyle/.style={
		for tree={circle,draw,font=\large}
	}
}

% disponi algoritmi
\usepackage{algorithm}
\usepackage{algorithmic}
\makeatletter
\renewcommand{\ALG@name}{Algoritmo}
\makeatother

% disponi numeri di pagina
\usepackage{fancyhdr}
\fancyhf{} 
\fancyfoot[L]{\sffamily{\thepage}}

\makeatletter
\fancyhead[L]{\raisebox{1ex}[0pt][0pt]{\sffamily{\@title \ \@date}}} 
\fancyhead[R]{\raisebox{1ex}[0pt][0pt]{\sffamily{\@author}}}
\makeatother

\begin{document}
% sezione (data)
\section{Lezione del 27-09-24}

% stili pagina
\thispagestyle{empty}
\pagestyle{fancy}

% testo
\subsubsection{Resistenza e cortocircuito in parallelo}
Poniamo di avere la configurazione:

\begin{center}
\begin{circuitikz}
    \draw (2,0) 
				to[short] (0,0) node[left] {$A$};
    
    \draw (2,-2) 
				to[short] (0,-2) node[left] {$B$};
		
		\draw (2,0 )to[R, l=$R$] (2,-2);
		\draw (4,0 )to[short] (4,-2);
		
		    
    % Draw horizontal lines at the top and bottom
    \draw (0,0) -- (4,0);
    \draw (0,-2) -- (4,-2);
\end{circuitikz}
\end{center}

Dove un resistore è in parallelo ad un corto circuito.
Intuitivamente, tutta la corrente passerà dal cortocircuito, e non dalla resistenza.
Possiamo modellizzare questo fatto in due modi:
\begin{itemize}
	\item Attraverso la formula per le resistenze in parallelo, avremo che:
		$$
		R_{eq} = \left( \frac{1}{R_1} + \frac{1}{R_2} \right)^{-1} = \frac{R_1R_2}{R_1 + R_2}, \quad R_1 = 0 \Rightarrow R_{eq} = \frac{0}{R_2} = 0
		$$	
		ergo resistenza nulla.
		
		La prima trasformazione è necessaria in quanto rimuove i vincoli sul dominio di $R_1$ e $R_2$ (che altrimenti non potrebbero essere 0).
	\item Notiamo che A e B sono effettivamente allo stesso potenziale, ergo abbiamo differenza di potenziale $V_{AB} = 0$ ai capi della resistenza.
		Applicando quindi la prima legge di Ohm $V_{AB} = i(t)R$ si ha $i(t) = 0$, cioè corrente costante nulla sulla resistenza.
\end{itemize}

\subsection{Altre configurazioni di resistenze}
Esistono altri modi di configurare le resistenze, che permettono di studiare circuiti su cui i metodi studiati finora non funzionano.

\subsubsection{Resistenze a triangolo}
Nelle resistenze a triangolo, una singola maglia di 3 nodi forma un triangolo con i lati 3 resistenze:

\begin{center}
\begin{circuitikz}
    \draw (1.73,1) node[right] {$1$}
				to[R, l=$R_{12}$] (-1.73,1) ;

    \draw (0,-2) 
				to[R, l=$R_{23}$] (-1.73,1) node[left] {$2$};

    \draw (1.73,1) 
				to[R, l=$R_{13}$] (0,-2) node[below] {$3$};

\end{circuitikz}
\end{center}

\subsubsection{Resistenze a stella}
Nelle resistenze a stella, più resistenze vengono collegate, da un'estremo, ad un singolo nodo centrale:

\begin{center}
\begin{circuitikz}
    \draw (0,0) 
				to[R, l=$R_1$] (1.73,1) node[right] {$1$};

    \draw (-1.73,1) node[left] {$2$}
				to[R, l=$R_2$] (0,0);

    \draw (0,-2) node[below] {$3$}
				to[R, l=$R_3$] (0,0);

\end{circuitikz}
\end{center}

\par\smallskip
Si possono trasformare resistenze a triangolo in resistenze a stella aggiungendo un nodo centrale $O$ e collegandovi i 3 nodi già esistenti attraverso le resistenze interne:

\begin{theorem}{Resistenze da stella a triangolo}	
$$
R_1 = \frac{R_{12}R_{13}}{R_{12} + R_{13} + R_{23}}
$$
$$
R_2 = \frac{R_{12}R_{23}}{R_{12} + R_{13} + R_{23}}
$$
$$
R_3 = \frac{R_{22}R_{13}}{R_{12} + R_{13} + R_{23}}
$$
\end{theorem}

Allo stesso modo, si possono trasformare resistenze a stella in resistenze a triangolo unendo i nodi fra di loro attraverso le resistenze esterne:

\begin{theorem}{Resistenze da triangolo a stella}	
$$
R_{12} = \frac{R_1R_2 + R_1R_3 + R_2R_3}{R_3}
$$
$$
R_{13} = \frac{R_1R_2 + R_1R_3 + R_2R_3}{R_2}
$$
$$
R_{22} = \frac{R_1R_2 + R_1R_3 + R_2R_3}{R_1}
$$
\end{theorem}

\subsection{Algoritmo per la resistenza equivalente}
A questo punto, si possono semplificare circuiti di resistori arbitrari applicando l'algoritmo:
\begin{algorithm}
\caption{Calcolo della resistenza equivalente}
\begin{algorithmic}
	\WHILE{ci sono $>1$ resistenze}

	\STATE Semplificare le resistenze in serie
	\STATE Semplificare le resistenze in parallelo
	
	\STATE Se non hai semplificato niente, trasforma un triangolo in stella o viceversa.

	\ENDWHILE
\end{algorithmic}
\end{algorithm}

La resistenza equivalente è a volte detta anche \textit{resistenza vista}. 
Questo perchè l'intero circuito si comporterà, per una qualsiasi rete esterna, come un singolo resistore di resistenza $R_{eq}$, ovvero avrà la stessa \textbf{risposta} di un singolo resistore di resistenza $R_{eq}$.
Analiticamente, questo significa che la funzione $f$ in $v(t) = f(i(t))$ (o la sua inversa) sono uguali per i due circuiti.
\end{document}
