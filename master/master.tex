
\documentclass[a4paper,11pt]{article}
\usepackage[a4paper, margin=8em]{geometry}

% usa i pacchetti per la scrittura in italiano
\usepackage[french,italian]{babel}
\usepackage[T1]{fontenc}
\usepackage[utf8]{inputenc}
\frenchspacing 

% usa i pacchetti per la formattazione matematica
\usepackage{amsmath, amssymb, amsthm, amsfonts}

% usa altri pacchetti
\usepackage{gensymb}
\usepackage{hyperref}
\usepackage{standalone}

% imposta il titolo
\title{Appunti Elettrotecnica}
\author{Luca Seggiani}
\date{2024}

% imposta lo stile
% usa helvetica
\usepackage[scaled]{helvet}
% usa palatino
\usepackage{palatino}
% usa un font monospazio guardabile
\usepackage{lmodern}

\renewcommand{\rmdefault}{ppl}
\renewcommand{\sfdefault}{phv}
\renewcommand{\ttdefault}{lmtt}

% disponi il titolo
\makeatletter
\renewcommand{\maketitle} {
	\begin{center} 
		\begin{minipage}[t]{.8\textwidth}
			\textsf{\huge\bfseries \@title} 
		\end{minipage}%
		\begin{minipage}[t]{.2\textwidth}
			\raggedleft \vspace{-1.65em}
			\textsf{\small \@author} \vfill
			\textsf{\small \@date}
		\end{minipage}
		\par
	\end{center}

	\thispagestyle{empty}
	\pagestyle{fancy}
}
\makeatother

% disponi teoremi
\usepackage{tcolorbox}
\newtcolorbox[auto counter, number within=section]{theorem}[2][]{%
	colback=blue!10, 
	colframe=blue!40!black, 
	sharp corners=northwest,
	fonttitle=\sffamily\bfseries, 
	title=~\thetcbcounter: #2, 
	#1
}

% disponi definizioni
\newtcolorbox[auto counter, number within=section]{definition}[2][]{%
	colback=red!10,
	colframe=red!40!black,
	sharp corners=northwest,
	fonttitle=\sffamily\bfseries,
	title=~\thetcbcounter: #2,
	#1
}

% U.D.M
\newcommand{\amp}{\ensuremath{\mathrm{A}}}
\newcommand{\volt}{\ensuremath{\mathrm{V}}}
\newcommand{\meter}{\ensuremath{\mathrm{m}}}
\newcommand{\second}{\ensuremath{\mathrm{s}}}
\newcommand{\farad}{\ensuremath{\mathrm{F}}}
\newcommand{\henry}{\ensuremath{\mathrm{H}}}
\newcommand{\siemens}{\ensuremath{\mathrm{S}}}

% circuiti
\usepackage{circuitikz}
\usetikzlibrary{babel}

% disegni
\usepackage{pgfplots}
\pgfplotsset{width=10cm,compat=1.9}

% disponi codice
\usepackage{listings}
\usepackage[table]{xcolor}

\lstdefinestyle{codestyle}{
		backgroundcolor=\color{black!5}, 
		commentstyle=\color{codegreen},
		keywordstyle=\bfseries\color{magenta},
		numberstyle=\sffamily\tiny\color{black!60},
		stringstyle=\color{green!50!black},
		basicstyle=\ttfamily\footnotesize,
		breakatwhitespace=false,         
		breaklines=true,                 
		captionpos=b,                    
		keepspaces=true,                 
		numbers=left,                    
		numbersep=5pt,                  
		showspaces=false,                
		showstringspaces=false,
		showtabs=false,                  
		tabsize=2
}

\lstdefinestyle{shellstyle}{
		backgroundcolor=\color{black!5}, 
		basicstyle=\ttfamily\footnotesize\color{black}, 
		commentstyle=\color{black}, 
		keywordstyle=\color{black},
		numberstyle=\color{black!5},
		stringstyle=\color{black}, 
		showspaces=false,
		showstringspaces=false, 
		showtabs=false, 
		tabsize=2, 
		numbers=none, 
		breaklines=true
}

\lstdefinelanguage{javascript}{
	keywords={typeof, new, true, false, catch, function, return, null, catch, switch, var, if, in, while, do, else, case, break},
	keywordstyle=\color{blue}\bfseries,
	ndkeywords={class, export, boolean, throw, implements, import, this},
	ndkeywordstyle=\color{darkgray}\bfseries,
	identifierstyle=\color{black},
	sensitive=false,
	comment=[l]{//},
	morecomment=[s]{/*}{*/},
	commentstyle=\color{purple}\ttfamily,
	stringstyle=\color{red}\ttfamily,
	morestring=[b]',
	morestring=[b]"
}

% disponi sezioni
\usepackage{titlesec}

\titleformat{\section}
	{\sffamily\Large\bfseries} 
	{\thesection}{1em}{} 
\titleformat{\subsection}
	{\sffamily\large\bfseries}   
	{\thesubsection}{1em}{} 
\titleformat{\subsubsection}
	{\sffamily\normalsize\bfseries} 
	{\thesubsubsection}{1em}{}

% disponi alberi
\usepackage{forest}

\forestset{
	rectstyle/.style={
		for tree={rectangle,draw,font=\large\sffamily}
	},
	roundstyle/.style={
		for tree={circle,draw,font=\large}
	}
}

% disponi algoritmi
\usepackage{algorithm}
\usepackage{algorithmic}
\makeatletter
\renewcommand{\ALG@name}{Algoritmo}
\makeatother

% disponi numeri di pagina
\usepackage{fancyhdr}
\fancyhf{} 
\fancyfoot[L]{\sffamily{\thepage}}

\makeatletter
\fancyhead[L]{\raisebox{1ex}[0pt][0pt]{\sffamily{\@title \ \@date}}} 
\fancyhead[R]{\raisebox{1ex}[0pt][0pt]{\sffamily{\@author}}}
\makeatother

\begin{document}

\pagestyle{fancy}
\thispagestyle{empty}
\renewcommand{\thispagestyle}[1]{}

\maketitle
\documentclass[a4paper,11pt]{article}
\usepackage[a4paper, margin=8em]{geometry}

% usa i pacchetti per la scrittura in italiano
\usepackage[french,italian]{babel}
\usepackage[T1]{fontenc}
\usepackage[utf8]{inputenc}
\frenchspacing 

% usa i pacchetti per la formattazione matematica
\usepackage{amsmath, amssymb, amsthm, amsfonts}

% usa altri pacchetti
\usepackage{gensymb}
\usepackage{hyperref}
\usepackage{standalone}

% imposta il titolo
\title{Appunti Elettrotecnica}
\author{Luca Seggiani}
\date{25-09-24}

% imposta lo stile
% usa helvetica
\usepackage[scaled]{helvet}
% usa palatino
\usepackage{palatino}
% usa un font monospazio guardabile
\usepackage{lmodern}

\renewcommand{\rmdefault}{ppl}
\renewcommand{\sfdefault}{phv}
\renewcommand{\ttdefault}{lmtt}

% disponi teoremi
\usepackage{tcolorbox}
\newtcolorbox[auto counter, number within=section]{theorem}[2][]{%
	colback=blue!10, 
	colframe=blue!40!black, 
	sharp corners=northwest,
	fonttitle=\sffamily\bfseries, 
	title=Teorema~\thetcbcounter: #2, 
	#1
}


% disegni
\usepackage{pgfplots}
\pgfplotsset{width=10cm,compat=1.9}

% disponi definizioni
\newtcolorbox[auto counter, number within=section]{definition}[2][]{%
	colback=red!10,
	colframe=red!40!black,
	sharp corners=northwest,
	fonttitle=\sffamily\bfseries,
	title=Definizione~\thetcbcounter: #2,
	#1
}

% disponi codice
\usepackage{listings}
\usepackage[table]{xcolor}

\lstdefinestyle{codestyle}{
		backgroundcolor=\color{black!5}, 
		commentstyle=\color{codegreen},
		keywordstyle=\bfseries\color{magenta},
		numberstyle=\sffamily\tiny\color{black!60},
		stringstyle=\color{green!50!black},
		basicstyle=\ttfamily\footnotesize,
		breakatwhitespace=false,         
		breaklines=true,                 
		captionpos=b,                    
		keepspaces=true,                 
		numbers=left,                    
		numbersep=5pt,                  
		showspaces=false,                
		showstringspaces=false,
		showtabs=false,                  
		tabsize=2
}

\lstdefinestyle{shellstyle}{
		backgroundcolor=\color{black!5}, 
		basicstyle=\ttfamily\footnotesize\color{black}, 
		commentstyle=\color{black}, 
		keywordstyle=\color{black},
		numberstyle=\color{black!5},
		stringstyle=\color{black}, 
		showspaces=false,
		showstringspaces=false, 
		showtabs=false, 
		tabsize=2, 
		numbers=none, 
		breaklines=true
}

\lstdefinelanguage{javascript}{
	keywords={typeof, new, true, false, catch, function, return, null, catch, switch, var, if, in, while, do, else, case, break},
	keywordstyle=\color{blue}\bfseries,
	ndkeywords={class, export, boolean, throw, implements, import, this},
	ndkeywordstyle=\color{darkgray}\bfseries,
	identifierstyle=\color{black},
	sensitive=false,
	comment=[l]{//},
	morecomment=[s]{/*}{*/},
	commentstyle=\color{purple}\ttfamily,
	stringstyle=\color{red}\ttfamily,
	morestring=[b]',
	morestring=[b]"
}

% disponi sezioni
\usepackage{titlesec}

\titleformat{\section}
	{\sffamily\Large\bfseries} 
	{\thesection}{1em}{} 
\titleformat{\subsection}
	{\sffamily\large\bfseries}   
	{\thesubsection}{1em}{} 
\titleformat{\subsubsection}
	{\sffamily\normalsize\bfseries} 
	{\thesubsubsection}{1em}{}

% disponi alberi
\usepackage{forest}

\forestset{
	rectstyle/.style={
		for tree={rectangle,draw,font=\large\sffamily}
	},
	roundstyle/.style={
		for tree={circle,draw,font=\large}
	}
}

% disponi algoritmi
\usepackage{algorithm}
\usepackage{algorithmic}
\makeatletter
\renewcommand{\ALG@name}{Algoritmo}
\makeatother

% disponi numeri di pagina
\usepackage{fancyhdr}
\fancyhf{} 
\fancyfoot[L]{\sffamily{\thepage}}

\makeatletter
\fancyhead[L]{\raisebox{1ex}[0pt][0pt]{\sffamily{\@title \ \@date}}} 
\fancyhead[R]{\raisebox{1ex}[0pt][0pt]{\sffamily{\@author}}}
\makeatother

\begin{document}
% sezione (data)
\section{Lezione del 25-09-24}

% stili pagina
\thispagestyle{empty}
\pagestyle{fancy}

% testo
\subsection{Introduzione}
Il corso di elettrotecnica riguarda lo studio dei \textbf{circuiti elettrici} e dei \textbf{macchinari elettrici}.

\subsubsection{Analisi dei circuiti elettrici}

Le leggi di Maxwell vanno a descrivere come si evolvono, nel tempo e nello spazio, i campi elettrici e magnetici.
Purtroppo,  le equazioni di Maxwell sono equazioni differenziali e legate fra di loro, ergo si possono spesso avere solo soluzioni numeriche.
Esistono però casì particolari in cui si possono fare semplificazioni considerevoli.

Un \textbf{circuito elettrico} è formato da fili conduttori e \textbf{componenti circuitali}.
All'interno di un circuito si va a descrivere un'onda elettrica:

$$
	\psi(s,t)
$$

rappresentata come una funzione di spazio e tempo.
Poniamo ad esempio la funzione, sulla sola posizione x:

$$
	\psi(x, t) = y \sin{\left( \frac{2\pi}{\lambda}x - \frac{2\pi}{T} t \right)}
$$

Questa funzione ha comunque due variabili: la posizione $x$ e il tempo $t$.
Immaginiamo di prendere un punto $x_0$ sul circuito elettrico:

$$
	\psi(t) = y \sin{\left( \frac{2\pi}{\lambda}x_0 - \frac{2\pi}{T} t \right)}
$$

Con $x_0 = 0$, annulliamo il primo termine.
A questo punto abbiamo ottenuto una funzione in una sola variabile:

$$
\psi(x_0, t) = y \sin{\left( - \frac{2\pi}{T} t \right)}
$$

ovvero una sinusoide invertita che oscilla fra un massimo di $y$ e un minimo di $-y$.

Questo significa che, mettendoci sul punto $x_0 = 0$ del circuito elettrico, notiamo che il valore dell'onda elettrica varia nel tempo seguendo questa funzione sinousidale.

Possiamo fare il processo invrso: fissiamo il tempo $t$, e vediamo come varia l'onda elettrica su diverse posizioni $x$ nel circuito.
Abbiamo, simbolicamente:

$$
	\psi(x) = y \sin{\left( \frac{2\pi}{\lambda}x - \frac{2\pi}{T} t_0 \right)}
$$

da cui ricaviamo l'equazione in una sola variabile $t$:

$$
	\psi(x) = y \sin{\left( \frac{2\pi}{\lambda}x \right)}
$$

ovvero una sinusoide che, come prima, oscilla fra un massimo di $y$ e un minimo di $-y$.
Si riporta un grafico:

\begin{center}
\begin{tikzpicture}
    \begin{axis}[
        xlabel={$x$},
        ylabel={$\psi(x)$},
        domain=-10:10, % set the x range you want
				samples=100,
        grid=major, % add a grid
				ytick={-2, 2},
				yticklabels={$-y$, $y$},
				ymin = -3, ymax = 3,
				width=15cm,
				height=7cm
    ]
    \addplot[
        blue,
        thick
    ] {2 * sin(50*x)}; 
    \end{axis}
\end{tikzpicture}
\end{center}

Questo significa che, all'istante $t_0 = 0$ notiamo che il valore dell'onda elettrica varia sulla lunghezza del circuito seguendo ancora questa funzione sinousidale.

Possiamo provare a calcolare lunghezza d'onda e periodo di questa oscillazione: visto che il periodo del seno è $2\pi$, abbiamo che nello spazio la lunghezza d'onda è $\lambda$ e nel tempo il periodo è $T$.

Proviamo a calcolare $\lambda$: sappiamo che la lunghezza d'onda equivale alla velocità di propagazione sulla frequenza dell'oscillazione, ovvero:

$$
\lambda = \frac{v}{f}
$$

Posti i valori $300 \cdot 10^6 \ \mathrm{m/s}$ per $v$ e $50 \ \text{Hz}$ per $f$ (la frequenza della rete elettrica), abbiamo:

$$
\lambda = \frac{3.00 \cdot 10^6 \ \mathrm{m/s}}{50 \ \text{Hz}} = 6000 \ \text{km} 
$$

Questa lunghezza d'onda diventa rilevante in trasmissioni elettriche su larga scala.

Possiamo fare considerazioni diverse se prendiamo in esempio le comunicazioni radio: lì si parla di frequenze $f >> 50 \ \text{Hz}$, nell'ordine dei megahertz o gigahertz.

L'elevata velocità della corrente ci permette di fare un'importante approssimazione e considerare \textbf{circuiti a parametri concetrati}.
Quest'ipotesi, in inglese \textit{lumped element model}, ci permette di ignorare l'estensione fisica del circuito, e quindi le variazioni delle funzioni d'onda sulla variabille spazio $s$, concentradosi sulla variabile tempo $t$.

\subsection{Grandezze}
Si usano le seguenti grandezze:
\subsubsection{Intensità di corrente}

\begin{definition}{Corrente elettrica}	
Si indica con $I$ la corrente elettrica, misurata in Ampere [$\mathrm{A}$], e definita come la variazione di carica:
$$
I = \frac{dq}{dt}
$$
\end{definition}

Si prende come positivo il verso in cui si muovono i portatori di carica positive, anche se sappiamo nella stragrande maggioranza dei casi i portatori di carica essere negativi, e quindi il movimento vero e proprio degli elettroni in direzione opposta.

Notiamo che se un segmento di circuito da $A$ a $B$ si ha una corrente $I_{AB}$, vale:

$$
I_{AB} = -I_{BA}
$$

\subsubsection{Differenza di potenziale}

\begin{definition}{Differenza di potenziale}
Si indica con $V$ la differenza di potenziale o \textit{tensione}, misurata in Volt [$\mathrm{V}$], e definita come il lavoro necessario a spostare una carica elementare positiva da un punto $A$ ad un punto $B$ sulla carica:

$$
	V_{AB}(t) = \frac{L_{AB}(t)}{q(t)}
$$
\end{definition}

Il segno del potenziale è definito come \textit{positivo} quando si deve vincere il campo magnetico per spingere la carica, ergo il campo elettrico svolge lavoro \textit{negativo} sulla carica.
Come prima, su segmenti di circuito da $A$ a $B$ vale:

$$
V_{AB} = -V_{BA}
$$

\subsubsection{Riferimenti associati e non associati}
I componenti circuitali, presi a sé, vengono detti \textbf{dipoli elettrici}, dal fatto che hanno 2 poli.
Di un dipolo elettrico si può misurare la differenza di potenziale ai capi e la corrente che vi scorre attraverso.

Quando si parla di tensione, o si parla di differenze di potenziale, o si assume un riferimento (lo zero del potenziale).
Non possiamo sapere a priori se il potenziale al capo di un dipolo è maggiore del potenziale all'altro capo: bisogna prima scegliere una direzione e poi vedere se il segno ricavato è concorde o meno con la nostra scelta.

Lo stesso vale per la corrente.
I riferimenti concordi al verso della corrente si dicono \textbf{associati}, quelli discordi si dicono \textbf{non associati}.

\subsubsection{Potenza elettrica}

\begin{definition}{Potenza elettrica}
Si indica con $P$ la potenza elettrica, misurata in Watt [$\mathrm{W}$] e definita come il prodotto:
$$
	P = IV
$$
fra corrente e tensione.
\end{definition}

Anche la potenza ha un segno, che in questo caso si riferisce a potenza \textit{erogata} o \textit{dissipata}.
La potenza calcolata sui riferimenti associati positiva è dissipata, quella negativa è erogata.
Viceversa, la potenza calcolata sui riferimenti non associati positiva è erogata, quella negativa è dissipata.

\subsubsection{Energia}

\begin{definition}{Energia}
Si indica con $W$ (non Watt!) l'energia, misurata in Joule [$\mathrm{J}$], o in Kilowatt/ora ($\mathrm{KW/h}$), e definita come l'integrale sul tempo della potenza:
$$
W = \int_{-\infty}^t P \ dt 
$$
\end{definition}

\subsection{Leggi di Kirchoff}
Iniziamo col dare dei nomi a particolari punti del circuito elettrico: i punti di incontro di più fili prendono il nome di \textbf{nodi}, e i fili che collegano i dipoli ai nodi prendono il nome di \textbf{rami}.
Da questo abbiamo che nei nodi si incontrano 3 o più rami.

Da qui possiamo definire la legge:
\begin{theorem}{Prima legge di Kirchoff}
		La somma algebrica delle correnti dei rami tagliati da una linea chiusa è uguale a 0.
		In particolare, la somma algebrica delle correnti entranti e uscenti da un nodo è uguale a 0.
\end{theorem}

Definiamo quindi il concetto di \textbf{maglia}: una maglia è un percorso chiuso di nodi e rami, ovvero un sottoinsieme di rami tali per cui spostandosi da un nodo all'altro si percorre ogni ramo una sola volta.
Sulle maglie si ha:

\begin{theorem}{Seconda legge di Kirchoff}
	La somma algebrica delle cadute di potenziale lungo una maglia è uguale a 0.
\end{theorem}

\end{document}


\documentclass[a4paper,11pt]{article}
\usepackage[a4paper, margin=8em]{geometry}

% usa i pacchetti per la scrittura in italiano
\usepackage[french,italian]{babel}
\usepackage[T1]{fontenc}
\usepackage[utf8]{inputenc}
\frenchspacing 

% usa i pacchetti per la formattazione matematica
\usepackage{amsmath, amssymb, amsthm, amsfonts}

% usa altri pacchetti
\usepackage{gensymb}
\usepackage{hyperref}
\usepackage{standalone}

% imposta il titolo
\title{Appunti Elettrotecnica}
\author{Luca Seggiani}
\date{2024}

% imposta lo stile
% usa helvetica
\usepackage[scaled]{helvet}
% usa palatino
\usepackage{palatino}
% usa un font monospazio guardabile
\usepackage{lmodern}

\renewcommand{\rmdefault}{ppl}
\renewcommand{\sfdefault}{phv}
\renewcommand{\ttdefault}{lmtt}

% disponi il titolo
\makeatletter
\renewcommand{\maketitle} {
	\begin{center} 
		\begin{minipage}[t]{.8\textwidth}
			\textsf{\huge\bfseries \@title} 
		\end{minipage}%
		\begin{minipage}[t]{.2\textwidth}
			\raggedleft \vspace{-1.65em}
			\textsf{\small \@author} \vfill
			\textsf{\small \@date}
		\end{minipage}
		\par
	\end{center}

	\thispagestyle{empty}
	\pagestyle{fancy}
}
\makeatother

% U.D.M
\newcommand{\amp}{\ensuremath{\mathrm{A}}}
\newcommand{\volt}{\ensuremath{\mathrm{V}}}
\newcommand{\meter}{\ensuremath{\mathrm{m}}}
\newcommand{\second}{\ensuremath{\mathrm{s}}}
\newcommand{\farad}{\ensuremath{\mathrm{F}}}
\newcommand{\henry}{\ensuremath{\mathrm{H}}}
\newcommand{\siemens}{\ensuremath{\mathrm{S}}}

% disponi teoremi
\usepackage{tcolorbox}
\newtcolorbox[auto counter, number within=section]{theorem}[2][]{%
	colback=blue!10, 
	colframe=blue!40!black, 
	sharp corners=northwest,
	fonttitle=\sffamily\bfseries, 
	title=Teorema~\thetcbcounter: #2, 
	#1
}

% disponi definizioni
\newtcolorbox[auto counter, number within=section]{definition}[2][]{%
	colback=red!10,
	colframe=red!40!black,
	sharp corners=northwest,
	fonttitle=\sffamily\bfseries,
	title=Definizione~\thetcbcounter: #2,
	#1
}

% disegni
\usepackage{pgfplots}
\pgfplotsset{width=10cm,compat=1.9}

% disponi codice
\usepackage{listings}
\usepackage[table]{xcolor}

\lstdefinestyle{codestyle}{
		backgroundcolor=\color{black!5}, 
		commentstyle=\color{codegreen},
		keywordstyle=\bfseries\color{magenta},
		numberstyle=\sffamily\tiny\color{black!60},
		stringstyle=\color{green!50!black},
		basicstyle=\ttfamily\footnotesize,
		breakatwhitespace=false,         
		breaklines=true,                 
		captionpos=b,                    
		keepspaces=true,                 
		numbers=left,                    
		numbersep=5pt,                  
		showspaces=false,                
		showstringspaces=false,
		showtabs=false,                  
		tabsize=2
}

\lstdefinestyle{shellstyle}{
		backgroundcolor=\color{black!5}, 
		basicstyle=\ttfamily\footnotesize\color{black}, 
		commentstyle=\color{black}, 
		keywordstyle=\color{black},
		numberstyle=\color{black!5},
		stringstyle=\color{black}, 
		showspaces=false,
		showstringspaces=false, 
		showtabs=false, 
		tabsize=2, 
		numbers=none, 
		breaklines=true
}

\lstdefinelanguage{javascript}{
	keywords={typeof, new, true, false, catch, function, return, null, catch, switch, var, if, in, while, do, else, case, break},
	keywordstyle=\color{blue}\bfseries,
	ndkeywords={class, export, boolean, throw, implements, import, this},
	ndkeywordstyle=\color{darkgray}\bfseries,
	identifierstyle=\color{black},
	sensitive=false,
	comment=[l]{//},
	morecomment=[s]{/*}{*/},
	commentstyle=\color{purple}\ttfamily,
	stringstyle=\color{red}\ttfamily,
	morestring=[b]',
	morestring=[b]"
}

% disponi sezioni
\usepackage{titlesec}

\titleformat{\section}
	{\sffamily\Large\bfseries} 
	{\thesection}{1em}{} 
\titleformat{\subsection}
	{\sffamily\large\bfseries}   
	{\thesubsection}{1em}{} 
\titleformat{\subsubsection}
	{\sffamily\normalsize\bfseries} 
	{\thesubsubsection}{1em}{}

% disponi alberi
\usepackage{forest}

\forestset{
	rectstyle/.style={
		for tree={rectangle,draw,font=\large\sffamily}
	},
	roundstyle/.style={
		for tree={circle,draw,font=\large}
	}
}

% disponi algoritmi
\usepackage{algorithm}
\usepackage{algorithmic}
\makeatletter
\renewcommand{\ALG@name}{Algoritmo}
\makeatother

% disponi numeri di pagina
\usepackage{fancyhdr}
\fancyhf{} 
\fancyfoot[L]{\sffamily{\thepage}}

\makeatletter
\fancyhead[L]{\raisebox{1ex}[0pt][0pt]{\sffamily{\@title \ \@date}}} 
\fancyhead[R]{\raisebox{1ex}[0pt][0pt]{\sffamily{\@author}}}
\makeatother

% circuiti
\usepackage{circuitikz}
\usetikzlibrary{babel}

\begin{document}
% sezione (data)
\section{Lezione del 26-09-24}

% stili pagina
\thispagestyle{empty}
\pagestyle{fancy}

% testo
\subsection{Dipolo elettrico}
Abbiamo introdotto i componenti circuitali come \textbf{dipoli elettrici}.
In particolare, diciamo che un dipolo elettrico è un componente, con una certa differenza di potenziale $V_{AB}$ ai suoi capi e una corrente $i_{AB}(t)$ che vi scorre all'interno, tale per cui si può definire una funzione del tipo:
$$
V_{AB} = f(i_{AB}(t))
$$

Possiamo individuare alcune caratteristiche importanti dei dipoli:
\begin{itemize}
	\item \textbf{Linearità:} un dipolo si dice lineare se la funzione che lega voltaggio e corrente è lineare.
		Tutti i dipoli che studieremo sono lineari (resistenze, capacitori, ecc...).
		Esistono però svariati dipoli che hanno risposte non lineari ai voltaggi/correnti a cui vengono sottoposti (diodi (risposte diverse a direzioni diverse della corrente), amplificatori operazionali, ecc...).
	\item \textbf{Tempo invarianza:} un dipolo si dice tempo invariante quando le sue caratteristiche non variano nel tempo.
	\item \textbf{Memoria:} un dipolo si dice dotato di memoria quando i suoi valori di corrente e tensione attuali dipendono da valori di corrente e tensioni ad un'istante $t$ precedente.
		I dipoli dotati di memoria presentano solitamente \textit{cicli di isteresi}.
	\item \textbf{Passività/attività:} si dice \textbf{passivo} un dipolo che dissipa potenza, e \textbf{attivo} un dipolo che la eroga.
		Più propriamente, si ha che un dipolo e passivo quando l'energia su di esso, presa un riferimento associato, è $\geq 0$.
\end{itemize}

\subsection{Resistori}
Un resistore è un componente circuitale caratterizzato dalla legge di Ohm ($ J = \sigma E$), e quindi formato da un materiale \textit{ohmico} che ha risposta lineare in densità di corrente alle variazioni del campo.
Si indica come:

\begin{center}
\begin{circuitikz}
\draw (0,0) to[ resistor ] (2,0); 
\end{circuitikz}
\end{center}

\begin{theorem}{Prima legge di Ohm}	
Il voltaggio è legato alla corrente, in un resistore, secondo la relazione:
$$V_R(t) = R \ i_R(t)$$
\end{theorem}

dove R prende il nome di \textbf{resistenza}, misurata in Ohm [$\ohm$], definita come:
$$
R = \frac{V}{i}
$$

\subsubsection{Resistenza e resistività}
Conosciamo la legge di Ohm sui materiali ohmici riportata prima.
Da questa legge si ricava:
\begin{theorem}{Seconda legge di Ohm}
	La resistenza di un filo di lunghezza $l$ e sezione $s$ è data da:
	$$
		R = \rho \frac{l}{s}
	$$
\end{theorem}
dove $\rho$ prende il nome di \textbf{resistività}, misurata in Ohm per metro [$\ohm \cdot \meter$].

Questo significa che la resistenza cresce con il crescere della lunghezza, e diminuisce con il crescere della sezione.

In verità questa non sono le uniche caratteristiche che influenzano la resistenza: un apporto significativo è dato anche dalla \textbf{temperatura}, alla quale la resistenza ha proporzionalità quasi lineare, ma che noi ignoreremo.

\subsubsection{Conduttanza e conducibilità}
Conviene definire altre due unità di misura: l'inverso della resistenza, detta \textbf{conduttanza}, che si misura in Siemens [$\ohm^{-1} = \siemens$], o in \textbf{mho} [$\mho = \ohm^{-1}$]:
$$
G = \frac{1}{R}
$$
e l'inverso della resistività, detta \textbf{conducibilità}, che si misura in [$\ohm^{-1} \cdot \meter^{-1}$]:
$$
\sigma = \frac{1}{\rho}
$$

\par\smallskip

I resistori sono inoltre:
\begin{itemize}
	\item Tempo invarianti (a patto di trascurare la temperatura);
	\item Senza memoria;
	\item Passivi (dissipano potenza per \textbf{effetto Joule}).
		Ciò si può dimostrare calcolando la potenza dalla prima legge di Ohm:
		$$
		p(t) = v_{AB}(t) \cdot i_{AB}(t) = R \ i_{AB}^2(t) \geq 0
		$$
		e calcolando l'energia come integrale:
		$$
		w(t) = \int_{-\infty}^{t} p(t)dt \Rightarrow w(t) > 0
		$$
\end{itemize}

\subsubsection{Circuiti aperti/chiusi}
Le resistenza, sopratutto nei loro casi limite, aiutano a modellizzare varie parti di un circuito:
\begin{itemize}
	\item \textbf{Cortocircuito:} indicato da una resistenza nulla, ergo:
		$$
			V_{AB}(t) = 0 \Leftrightarrow R = 0
		$$
		
		Modellizza il filo ideale, ergo ciò che per noi è un ramo.
	\item \textbf{Circuito aperto:} indicato da una corrente nulla, ergo:
		$$
			i_{AB} = 0 \Leftrightarrow R = +\infty
		$$

		Modellizza interruzioni nel circuito: si può dimostrare che la corrente attraverso un'interruzione in un circuito è nulla sfruttando la prima legge di Kirchoff: una linea chiusa che comprende il nodo finale di un'interruzione avrà un ramo entrante e 0 uscenti, ovvero corrente entrante nulla.
\end{itemize}

\subsubsection{Resistenze in serie}
Poniamo di avere una configurazione di resistenze del tipo:

\begin{center}
\begin{circuitikz}
    \draw (0,0) node[left] {$A$} 
        to[R, l=$R_1$] (2,0) 
        to[R, l=$R_2$] (4,0) 
        to[short] (5,0)
        node[anchor=west,xshift=0.15cm] {\dots} (6,0) 
        to[short] (7,0)
        to[R, l=$R_n$] (9,0) node[right] {$B$};

    \draw[decorate,decoration={brace,amplitude=10pt,mirror}] (0,-1) -- (9,-1)
        node[midway,below=10pt]{$n$ resistori};
\end{circuitikz}
\end{center}

Vogliamo calcolare una resistenza $R_{eq}$ che valga quando la resistenza cumulativa di tutte e $n$ le resistenze.
Abbiamo allora che la corrente lungo ogni resistenza $i(t)$ è costante, mentre ogni resistenza contribuisce al potenziale $V_{AB}$ con una certa caduta di potenziale $V_1(t), V_2(t), ..., V_n(T)$.
Si applica quindi la prima legge di Ohm:
$$
V_{AB} = V_1{t} + V_2{t} + ... + V_n{t} = R_1 \cdot i(t) + R_2 \cdot i(t) + ... + R_n \cdot i(t) = i(t) \cdot \left( R_1 + R_2 + ... + R_n \right)
$$
quindi, da $V_{AB} = i(t) \ R_{eq}$ si ha:

\begin{theorem}{Resistenze in serie}	
$$
R_{eq} = R_1 + R_2 + ... + R_n
$$
\end{theorem}

\subsubsection{Resistenze in parallelo}
Poniamo di avere le resistenze in parallelo invece che in serie:
\begin{center}
\begin{circuitikz}
    \draw (2,0) 
				to[short] (0,0) node[left] {$A$};
    
    \draw (2,-2) 
				to[short] (0,-2) node[left] {$B$};
		
		\draw (2,0 )to[R, l=$R_1$] (2,-2);
		\draw (4,0 )to[R, l=$R_2$] (4,-2);
		
		
    \draw (4,0) node[anchor=west,xshift=0.7cm] {\dots} (6,0);
    \draw (4,-2) node[anchor=west,xshift=0.7cm] {\dots} (6,-2); 
		
		\draw (8,0 )to[R, l=$R_n$] (8,-2);
        
    % Draw horizontal lines at the top and bottom
    \draw (0,0) -- (4,0);
    \draw (0,-2) -- (4,-2);

    \draw (6,0) -- (8,0);
    \draw (6,-2) -- (8,-2);

    % Bracket below the resistors
    \draw[decorate,decoration={brace,amplitude=10pt,mirror}] (0,-3) -- (8,-3)
        node[midway,below=10pt]{$n$ resistori};
\end{circuitikz}
\end{center}

Vogliamo ancora calcolare una resistenza $R_{eq}$ che valga quando la resistenza cumulativa di tutte e $n$ le resistenze.
Qui abbiamo che la differenza di potenziale lungo ogni resistenza $V(t)$ costante.
Si applica ancora la prima legge di Ohm:
$$
i = i_1(t) + i_2(t) + ... + i_n(t) = \frac{V_{AB}(t)}{R_1} + \frac{V_{AB}(t)}{R_2} + ... + \frac{V_{AB}(t)}{R_n}
$$
conviene raccogliere e passare alle conduttanze:
$$
G_{eq} = V_{AB}(t)(G_1 + G_2 + ... + G_n) = G_{eq} \cdot V_{AB}(t)
$$
Ora, se $G = \frac{1}{R}$:
$$
R_{eq} = G_{eq}^{-1} = \left( \frac{1}{R_1} + \frac{1}{R_2} + ... + \frac{1}{R_n} \right)^{-1} 
$$
quindi, si ha:

\begin{theorem}{Resistenze in parallelo}
$$
R_{eq} = \left( \frac{1}{R_1} + \frac{1}{R_2} + ... + \frac{1}{R_n} \right)^{-1}
$$
\end{theorem}

\end{document}


\documentclass[a4paper,11pt]{article}
\usepackage[a4paper, margin=8em]{geometry}

% usa i pacchetti per la scrittura in italiano
\usepackage[french,italian]{babel}
\usepackage[T1]{fontenc}
\usepackage[utf8]{inputenc}
\frenchspacing 

% usa i pacchetti per la formattazione matematica
\usepackage{amsmath, amssymb, amsthm, amsfonts}

% usa altri pacchetti
\usepackage{gensymb}
\usepackage{hyperref}
\usepackage{standalone}

% imposta il titolo
\title{Appunti Elettrotecnica}
\author{Luca Seggiani}
\date{2024}

% imposta lo stile
% usa helvetica
\usepackage[scaled]{helvet}
% usa palatino
\usepackage{palatino}
% usa un font monospazio guardabile
\usepackage{lmodern}

\renewcommand{\rmdefault}{ppl}
\renewcommand{\sfdefault}{phv}
\renewcommand{\ttdefault}{lmtt}

% disponi il titolo
\makeatletter
\renewcommand{\maketitle} {
	\begin{center} 
		\begin{minipage}[t]{.8\textwidth}
			\textsf{\huge\bfseries \@title} 
		\end{minipage}%
		\begin{minipage}[t]{.2\textwidth}
			\raggedleft \vspace{-1.65em}
			\textsf{\small \@author} \vfill
			\textsf{\small \@date}
		\end{minipage}
		\par
	\end{center}

	\thispagestyle{empty}
	\pagestyle{fancy}
}
\makeatother

% disponi teoremi
\usepackage{tcolorbox}
\newtcolorbox[auto counter, number within=section]{theorem}[2][]{%
	colback=blue!10, 
	colframe=blue!40!black, 
	sharp corners=northwest,
	fonttitle=\sffamily\bfseries, 
	title=~\thetcbcounter: #2, 
	#1
}

% disponi definizioni
\newtcolorbox[auto counter, number within=section]{definition}[2][]{%
	colback=red!10,
	colframe=red!40!black,
	sharp corners=northwest,
	fonttitle=\sffamily\bfseries,
	title=~\thetcbcounter: #2,
	#1
}

% U.D.M
\newcommand{\amp}{\ensuremath{\mathrm{A}}}
\newcommand{\volt}{\ensuremath{\mathrm{V}}}
\newcommand{\meter}{\ensuremath{\mathrm{m}}}
\newcommand{\second}{\ensuremath{\mathrm{s}}}
\newcommand{\farad}{\ensuremath{\mathrm{F}}}
\newcommand{\henry}{\ensuremath{\mathrm{H}}}
\newcommand{\siemens}{\ensuremath{\mathrm{S}}}

% circuiti
\usepackage{circuitikz}
\usetikzlibrary{babel}

% disegni
\usepackage{pgfplots}
\pgfplotsset{width=10cm,compat=1.9}

% disponi codice
\usepackage{listings}
\usepackage[table]{xcolor}

\lstdefinestyle{codestyle}{
		backgroundcolor=\color{black!5}, 
		commentstyle=\color{codegreen},
		keywordstyle=\bfseries\color{magenta},
		numberstyle=\sffamily\tiny\color{black!60},
		stringstyle=\color{green!50!black},
		basicstyle=\ttfamily\footnotesize,
		breakatwhitespace=false,         
		breaklines=true,                 
		captionpos=b,                    
		keepspaces=true,                 
		numbers=left,                    
		numbersep=5pt,                  
		showspaces=false,                
		showstringspaces=false,
		showtabs=false,                  
		tabsize=2
}

\lstdefinestyle{shellstyle}{
		backgroundcolor=\color{black!5}, 
		basicstyle=\ttfamily\footnotesize\color{black}, 
		commentstyle=\color{black}, 
		keywordstyle=\color{black},
		numberstyle=\color{black!5},
		stringstyle=\color{black}, 
		showspaces=false,
		showstringspaces=false, 
		showtabs=false, 
		tabsize=2, 
		numbers=none, 
		breaklines=true
}

\lstdefinelanguage{javascript}{
	keywords={typeof, new, true, false, catch, function, return, null, catch, switch, var, if, in, while, do, else, case, break},
	keywordstyle=\color{blue}\bfseries,
	ndkeywords={class, export, boolean, throw, implements, import, this},
	ndkeywordstyle=\color{darkgray}\bfseries,
	identifierstyle=\color{black},
	sensitive=false,
	comment=[l]{//},
	morecomment=[s]{/*}{*/},
	commentstyle=\color{purple}\ttfamily,
	stringstyle=\color{red}\ttfamily,
	morestring=[b]',
	morestring=[b]"
}

% disponi sezioni
\usepackage{titlesec}

\titleformat{\section}
	{\sffamily\Large\bfseries} 
	{\thesection}{1em}{} 
\titleformat{\subsection}
	{\sffamily\large\bfseries}   
	{\thesubsection}{1em}{} 
\titleformat{\subsubsection}
	{\sffamily\normalsize\bfseries} 
	{\thesubsubsection}{1em}{}

% disponi alberi
\usepackage{forest}

\forestset{
	rectstyle/.style={
		for tree={rectangle,draw,font=\large\sffamily}
	},
	roundstyle/.style={
		for tree={circle,draw,font=\large}
	}
}

% disponi algoritmi
\usepackage{algorithm}
\usepackage{algorithmic}
\makeatletter
\renewcommand{\ALG@name}{Algoritmo}
\makeatother

% disponi numeri di pagina
\usepackage{fancyhdr}
\fancyhf{} 
\fancyfoot[L]{\sffamily{\thepage}}

\makeatletter
\fancyhead[L]{\raisebox{1ex}[0pt][0pt]{\sffamily{\@title \ \@date}}} 
\fancyhead[R]{\raisebox{1ex}[0pt][0pt]{\sffamily{\@author}}}
\makeatother

\begin{document}
% sezione (data)
\section{Lezione del 27-09-24}

% stili pagina
\thispagestyle{empty}
\pagestyle{fancy}

% testo
\subsubsection{Resistenza e cortocircuito in parallelo}
Poniamo di avere la configurazione:

\begin{center}
\begin{circuitikz}
    \draw (2,0) 
				to[short] (0,0) node[left] {$A$};
    
    \draw (2,-2) 
				to[short] (0,-2) node[left] {$B$};
		
		\draw (2,0 )to[R, l=$R$] (2,-2);
		\draw (4,0 )to[short] (4,-2);
		
		    
    % Draw horizontal lines at the top and bottom
    \draw (0,0) -- (4,0);
    \draw (0,-2) -- (4,-2);
\end{circuitikz}
\end{center}

Dove un resistore è in parallelo ad un corto circuito.
Intuitivamente, tutta la corrente passerà dal cortocircuito, e non dalla resistenza.
Possiamo modellizzare questo fatto in due modi:
\begin{itemize}
	\item Attraverso la formula per le resistenze in parallelo, avremo che:
		$$
		R_{eq} = \left( \frac{1}{R_1} + \frac{1}{R_2} \right)^{-1} = \frac{R_1R_2}{R_1 + R_2}, \quad R_1 = 0 \Rightarrow R_{eq} = \frac{0}{R_2} = 0
		$$	
		ergo resistenza nulla.
		
		La prima trasformazione è necessaria in quanto rimuove i vincoli sul dominio di $R_1$ e $R_2$ (che altrimenti non potrebbero essere 0).
	\item Notiamo che A e B sono effettivamente allo stesso potenziale, ergo abbiamo differenza di potenziale $V_{AB} = 0$ ai capi della resistenza.
		Applicando quindi la prima legge di Ohm $V_{AB} = i(t)R$ si ha $i(t) = 0$, cioè corrente costante nulla sulla resistenza.
\end{itemize}

\subsection{Altre configurazioni di resistenze}
Esistono altri modi di configurare le resistenze, che permettono di studiare circuiti su cui i metodi studiati finora non funzionano.

\subsubsection{Resistenze a triangolo}
Nelle resistenze a triangolo, una singola maglia di 3 nodi forma un triangolo con i lati 3 resistenze:

\begin{center}
\begin{circuitikz}
    \draw (1.73,1) node[right] {$1$}
				to[R, l=$R_{12}$] (-1.73,1) ;

    \draw (0,-2) 
				to[R, l=$R_{23}$] (-1.73,1) node[left] {$2$};

    \draw (1.73,1) 
				to[R, l=$R_{13}$] (0,-2) node[below] {$3$};

\end{circuitikz}
\end{center}

\subsubsection{Resistenze a stella}
Nelle resistenze a stella, più resistenze vengono collegate, da un'estremo, ad un singolo nodo centrale:

\begin{center}
\begin{circuitikz}
    \draw (0,0) 
				to[R, l=$R_1$] (1.73,1) node[right] {$1$};

    \draw (-1.73,1) node[left] {$2$}
				to[R, l=$R_2$] (0,0);

    \draw (0,-2) node[below] {$3$}
				to[R, l=$R_3$] (0,0);

\end{circuitikz}
\end{center}

\par\smallskip
Si possono trasformare resistenze a triangolo in resistenze a stella aggiungendo un nodo centrale $O$ e collegandovi i 3 nodi già esistenti attraverso le resistenze interne:

\begin{theorem}{Resistenze da stella a triangolo}	
$$
R_1 = \frac{R_{12}R_{13}}{R_{12} + R_{13} + R_{23}}
$$
$$
R_2 = \frac{R_{12}R_{23}}{R_{12} + R_{13} + R_{23}}
$$
$$
R_3 = \frac{R_{22}R_{13}}{R_{12} + R_{13} + R_{23}}
$$
\end{theorem}

Allo stesso modo, si possono trasformare resistenze a stella in resistenze a triangolo unendo i nodi fra di loro attraverso le resistenze esterne:

\begin{theorem}{Resistenze da triangolo a stella}	
$$
R_{12} = \frac{R_1R_2 + R_1R_3 + R_2R_3}{R_3}
$$
$$
R_{13} = \frac{R_1R_2 + R_1R_3 + R_2R_3}{R_2}
$$
$$
R_{22} = \frac{R_1R_2 + R_1R_3 + R_2R_3}{R_1}
$$
\end{theorem}

\subsection{Algoritmo per la resistenza equivalente}
A questo punto, si possono semplificare circuiti di resistori arbitrari applicando l'algoritmo:
\begin{algorithm}
\caption{Calcolo della resistenza equivalente}
\begin{algorithmic}
	\WHILE{ci sono $>1$ resistenze}

	\STATE Semplificare le resistenze in serie
	\STATE Semplificare le resistenze in parallelo
	
	\STATE Se non hai semplificato niente, trasforma un triangolo in stella o viceversa.

	\ENDWHILE
\end{algorithmic}
\end{algorithm}

La resistenza equivalente è a volte detta anche \textit{resistenza vista}. 
Questo perchè l'intero circuito si comporterà, per una qualsiasi rete esterna, come un singolo resistore di resistenza $R_{eq}$, ovvero avrà la stessa \textbf{risposta} di un singolo resistore di resistenza $R_{eq}$.
Analiticamente, questo significa che la funzione $f$ in $v(t) = f(i(t))$ (o la sua inversa) sono uguali per i due circuiti.
\end{document}


\documentclass[a4paper,11pt]{article}
\usepackage[a4paper, margin=8em]{geometry}

% usa i pacchetti per la scrittura in italiano
\usepackage[french,italian]{babel}
\usepackage[T1]{fontenc}
\usepackage[utf8]{inputenc}
\frenchspacing 

% usa i pacchetti per la formattazione matematica
\usepackage{amsmath, amssymb, amsthm, amsfonts}

% usa altri pacchetti
\usepackage{gensymb}
\usepackage{hyperref}
\usepackage{standalone}

% imposta il titolo
\title{Appunti Elettrotecnica}
\author{Luca Seggiani}
\date{2024}

% imposta lo stile
% usa helvetica
\usepackage[scaled]{helvet}
% usa palatino
\usepackage{palatino}
% usa un font monospazio guardabile
\usepackage{lmodern}

\renewcommand{\rmdefault}{ppl}
\renewcommand{\sfdefault}{phv}
\renewcommand{\ttdefault}{lmtt}

% disponi il titolo
\makeatletter
\renewcommand{\maketitle} {
	\begin{center} 
		\begin{minipage}[t]{.8\textwidth}
			\textsf{\huge\bfseries \@title} 
		\end{minipage}%
		\begin{minipage}[t]{.2\textwidth}
			\raggedleft \vspace{-1.65em}
			\textsf{\small \@author} \vfill
			\textsf{\small \@date}
		\end{minipage}
		\par
	\end{center}

	\thispagestyle{empty}
	\pagestyle{fancy}
}
\makeatother

% disponi teoremi
\usepackage{tcolorbox}
\newtcolorbox[auto counter, number within=section]{theorem}[2][]{%
	colback=blue!10, 
	colframe=blue!40!black, 
	sharp corners=northwest,
	fonttitle=\sffamily\bfseries, 
	title=~\thetcbcounter: #2, 
	#1
}

% disponi definizioni
\newtcolorbox[auto counter, number within=section]{definition}[2][]{%
	colback=red!10,
	colframe=red!40!black,
	sharp corners=northwest,
	fonttitle=\sffamily\bfseries,
	title=~\thetcbcounter: #2,
	#1
}

% U.D.M
\newcommand{\amp}{\ensuremath{\mathrm{A}}}
\newcommand{\volt}{\ensuremath{\mathrm{V}}}
\newcommand{\meter}{\ensuremath{\mathrm{m}}}
\newcommand{\second}{\ensuremath{\mathrm{s}}}
\newcommand{\farad}{\ensuremath{\mathrm{F}}}
\newcommand{\henry}{\ensuremath{\mathrm{H}}}
\newcommand{\siemens}{\ensuremath{\mathrm{S}}}

% circuiti
\usepackage{circuitikz}
\usetikzlibrary{babel}

% disegni
\usepackage{pgfplots}
\pgfplotsset{width=10cm,compat=1.9}

% disponi codice
\usepackage{listings}
\usepackage[table]{xcolor}

\lstdefinestyle{codestyle}{
		backgroundcolor=\color{black!5}, 
		commentstyle=\color{codegreen},
		keywordstyle=\bfseries\color{magenta},
		numberstyle=\sffamily\tiny\color{black!60},
		stringstyle=\color{green!50!black},
		basicstyle=\ttfamily\footnotesize,
		breakatwhitespace=false,         
		breaklines=true,                 
		captionpos=b,                    
		keepspaces=true,                 
		numbers=left,                    
		numbersep=5pt,                  
		showspaces=false,                
		showstringspaces=false,
		showtabs=false,                  
		tabsize=2
}

\lstdefinestyle{shellstyle}{
		backgroundcolor=\color{black!5}, 
		basicstyle=\ttfamily\footnotesize\color{black}, 
		commentstyle=\color{black}, 
		keywordstyle=\color{black},
		numberstyle=\color{black!5},
		stringstyle=\color{black}, 
		showspaces=false,
		showstringspaces=false, 
		showtabs=false, 
		tabsize=2, 
		numbers=none, 
		breaklines=true
}

\lstdefinelanguage{javascript}{
	keywords={typeof, new, true, false, catch, function, return, null, catch, switch, var, if, in, while, do, else, case, break},
	keywordstyle=\color{blue}\bfseries,
	ndkeywords={class, export, boolean, throw, implements, import, this},
	ndkeywordstyle=\color{darkgray}\bfseries,
	identifierstyle=\color{black},
	sensitive=false,
	comment=[l]{//},
	morecomment=[s]{/*}{*/},
	commentstyle=\color{purple}\ttfamily,
	stringstyle=\color{red}\ttfamily,
	morestring=[b]',
	morestring=[b]"
}

% disponi sezioni
\usepackage{titlesec}

\titleformat{\section}
	{\sffamily\Large\bfseries} 
	{\thesection}{1em}{} 
\titleformat{\subsection}
	{\sffamily\large\bfseries}   
	{\thesubsection}{1em}{} 
\titleformat{\subsubsection}
	{\sffamily\normalsize\bfseries} 
	{\thesubsubsection}{1em}{}

% disponi alberi
\usepackage{forest}

\forestset{
	rectstyle/.style={
		for tree={rectangle,draw,font=\large\sffamily}
	},
	roundstyle/.style={
		for tree={circle,draw,font=\large}
	}
}

% disponi algoritmi
\usepackage{algorithm}
\usepackage{algorithmic}
\makeatletter
\renewcommand{\ALG@name}{Algoritmo}
\makeatother

% disponi numeri di pagina
\usepackage{fancyhdr}
\fancyhf{} 
\fancyfoot[L]{\sffamily{\thepage}}

\makeatletter
\fancyhead[L]{\raisebox{1ex}[0pt][0pt]{\sffamily{\@title \ \@date}}} 
\fancyhead[R]{\raisebox{1ex}[0pt][0pt]{\sffamily{\@author}}}
\makeatother

\begin{document}
% sezione (data)
\section{Lezione del 02-10-24}

% stili pagina
\thispagestyle{empty}
\pagestyle{fancy}

% testo
\subsection{Generatori}
I generatori sono i componenti che spostano le cariche attraverso le reti elettriche.
Dividiamo i generatori in due macrocategorie, in base alle loro caratteristiche:
\begin{itemize}
	\item \textbf{Indipendenti:} hanno sempre le stesse caratteristiche, e portano energia all'interno del circuito;
	\item \textbf{Dipendenti:} hanno caratteristiche \textit{pilotate} da altri fattori del circuito, non portano energia in esso e quindi non sono diversi dagli altri dipoli passivi già visti.
\end{itemize}

Inoltre dividiamo entrambe in altre due categorie, in base al tipo di operazione che svolgono:
\begin{itemize}
	\item \textbf{Generatori di tensione:} mantengono i loro capi a differenza di potenziale costante;
	\item \textbf{Generatori di corrente:} mantengono una corrente costante al loro interno.
\end{itemize}

Infine, dividiamo in due ulteriori modalità di operazione:
\begin{itemize}
	\item \textbf{Corrente continua:} mantengono la corrente costante. Si dicono C.C. (Corrente Continua), o D.C. (Direct Current). Il grafico della corrente sarà:
		\begin{center}
\begin{tikzpicture}
    \begin{axis}[
        xlabel={$t$},
        ylabel={$i(t)$},
        domain=-10:10, % set the x range you want
				samples=100,
        grid=major, % add a grid
				ytick={0, 2},
				yticklabels={0, I},
				ymin = -1, ymax = 3,
				width=14cm,
				height=7cm
    ]
    \addplot[
        blue,
        thick
    ] {2}; 
    \end{axis}
\end{tikzpicture}
\end{center}

	\item \textbf{Corrente alternata:} mantengono la corrente in regime sinousidale. Si dicono C.A. (Corrente Alternata), o A.C. (Alternating Current). Il grafico della corrente alternata è stato già visto all'inizio del corso, ha equazione: 
$$
	i(t) = A \sin{\left(\frac{2\pi}{T} t \right)}
$$
con $A$ ampiezza e $T$ periodo, e grafico:

\begin{center}
\begin{tikzpicture}
    \begin{axis}[
        xlabel={$t$},
        ylabel={$i(t)$},
        domain=-10:10, % set the x range you want
				samples=100,
        grid=major, % add a grid
				ytick={-2, 2},
				yticklabels={$-A$, $A$},
				ymin = -3, ymax = 3,
				width=14cm,
				height=7cm
    ]
    \addplot[
        blue,
        thick
    ] {2 * sin(50*x)}; 
    \end{axis}
\end{tikzpicture}
\end{center}
\end{itemize}

Esistono poi altri regimi di applicazione della corrente, che vedremo per casi specifici (impulsi, gradini, ecc...).

Riportiamo intanto ogni combinazione delle prime quattro tipologie nel dettaglio.

\subsubsection{Generatori di tensione}
Un generatore di tensione (o voltaggio) ideale è un componente circuitale che mantiene i suoi capi $A$ e $B$ ad una differenza di potenziale $V_{AB}$ costante, ovvero:
$$ v(t) = E(t) = V $$
dove con $E$ si indica la forza elettromotrice. 
Si indica come:

\begin{center}
\begin{circuitikz}
\draw (0,0) to[ european voltage source ] (2,0); 
\end{circuitikz}
\end{center}

Si nota che a voltaggio nullo, un generatore di tensione equivale a un corto circuito (un filo ideale).

\par\medskip
\noindent
\textbf{\textsf{Correlazione con la corrente}} \\
La tensione erogata da un generatore di tensione è costante, qualsiasi sia la corrente che lo attraversa:
$$ v(i) = \mathrm{const.} $$
Il grafico di correlazione corrente-voltaggio sarà quindi:

\begin{center}
\begin{tikzpicture}
    \begin{axis}[
        xlabel={$i$},
        ylabel={$v$},
        domain=0:4, % set the x range you want
				samples=100,
        grid=major, % add a grid
				ytick={0, 2},
				yticklabels={0, $V$},
				xtick={0},
				ymin = -0.5, ymax = 3,
				xmin = -0.5,
				width=15cm,
				height=7cm
    ]
    \addplot[
        blue,
        thick
    ] {2}; 

    \end{axis}
\end{tikzpicture}
\end{center}

\par\medskip
\noindent
\textbf{\textsf{Correlazione con la potenza}} \\
Tradizionalmente si descrivono i generatori di tensione attraverso riferimenti non associati di corrente e tensione.
Resta il fatto che la potenza:
$$ p(t) = v(t)i(t) = E(t)i(t) $$
quando è erogata dal generatore, è $> 0$.

\par\medskip
\noindent
\textbf{\textsf{Collegamenti in serie}} \\
Per sommare i contributi al voltaggio di più generatori di voltaggio, li disponiamo in serie:

\begin{center}
\begin{circuitikz}
    \draw (0,0) node[left] {$A$} 
        to[european voltage source, l=$V_1$] (2,0) 
        to[european voltage source, l=$V_2$] (4,0) 
        to[short] (5,0)
        node[anchor=west,xshift=0.15cm] {\dots} (6,0) 
        to[short] (7,0)
        to[european voltage source, l=$V_n$] (9,0) node[right] {$B$};

    \draw[decorate,decoration={brace,amplitude=10pt,mirror}] (0,-1) -- (9,-1)
        node[midway,below=10pt]{$n$ generatori};
\end{circuitikz}
\end{center}

Abbiamo che il contributo totale dei generatori equivale a quello di un singolo generatore $E_T$ di voltaggio:
$$ V_T = V_1 + V_2 + ... + V_n $$

\par\medskip
\noindent
\textbf{\textsf{Collegamenti in parallelo}} \\
Non si possono collegare generatori di voltaggio in parallelo, a meno che questi non abbiano lo stesso voltaggio (e quindi risultino in movimento nullo di carica):

\begin{center}
\begin{circuitikz}
    \draw (2,0) 
				to[short] (0,0) node[left] {$A$};
    
    \draw (2,-2) 
				to[short] (0,-2) node[left] {$B$};
		
		\draw (2,0 )to[european voltage source, l=$V_1$] (2,-2);
		\draw (4,0 )to[european voltage source, l=$V_2$] (4,-2);
		
        
    % Draw horizontal lines at the top and bottom
    \draw (0,0) -- (4,0);
    \draw (0,-2) -- (4,-2);
\end{circuitikz}
\end{center}

Dove si ha, dall'applicazione della seconda legge di Kirchoff:
$$
V_1 - V_2 = 0 \Rightarrow V_1 = V_2 
$$
che sarebbe altrimenti violata.

Nella realtà, se si provasse a collegare due generatori di tensione di voltaggio diverso in parallelo, questi proverebbero a imporre la loro differenza di potenziale sui due rami del circuito, creando forti correnti, e probabilmente causando danni termici ad esso o a loro stessi.


\subsubsection{Generatori di corrente}
Un generatore di corrente ideale è un componente circuitale che mantiene attraverso di sé una corrente costante, ovvero:
$$ i(t) = I $$
Si indica come:

\begin{center}
\begin{circuitikz}
\draw (0,0) to[ european current source ] (2,0); 
\end{circuitikz}
\end{center}

Si nota che a corrente nulla, un generatore di corrente equivale a un circuito aperto.

\par\medskip
\noindent
\textbf{\textsf{Correlazione con il voltaggio}} \\
Un generatore di corrente mantiene la stessa corrente qualsiasi sia il voltaggio.
$$ i(v) = \mathrm{const.} $$
Il grafico di correlazione corrente-voltaggio sarà quindi:

\begin{center}
\begin{tikzpicture}
    \begin{axis}[
        xlabel={$i$},
        ylabel={$v$},
        domain=0:4, % set the x range you want
				samples=100,
        grid=major, % add a grid
				ytick={0},
				yticklabels={0},
				xtick={0, 1.5},
				xticklabels={0, $I$},
				ymin = -0.5, ymax = 4,
				xmin = -0.25, xmax = 1.75,
				width=15cm,
				height=7cm
    ]
    \addplot[
        red,
        thick,
        domain=-0.5:3 % set the y range you want
    ] coordinates {(1.5,0) (1.5,3)};


    \end{axis}
\end{tikzpicture}
\end{center}

\par\medskip
\noindent
\textbf{\textsf{Correlazione con la potenza}} \\
Come per i generatori di tensione, si descrivono i generatori di corrente attraverso riferimenti non associati di corrente e tensione.
Resta comunque il fatto che la potenza:
$$ p(t) = v(t)i(t) = v(t)I(t) $$
quando è erogata dal generatore, è $> 0$.

\par\medskip
\noindent
\textbf{\textsf{Collegamenti in serie}} \\
Non si possono collegare generatori di corrente in serie, a meno che questi non abbiano la stessa carica (e quindi risultino in movimento uniforme di carica):

\begin{center}
\begin{circuitikz}
    \draw (0,0) node[left] {$A$} 
        to[european current source, l=$I_1$] (2,0) 
        to[european current source, l=$I_2$] (4,0) node[right] {$B$};
\end{circuitikz}
\end{center}

Dove si ha, dall'applicazione della prima legge di Kirchoff:
$$
I_1 - I_2 = 0 \Rightarrow I_1 = I_2 
$$
che sarebbe altrimenti violata.

Come prima, questa situazione non è effettivamente modellizzabile nella realtà usando il modello studiato.
In verità il generatore di corrente in sé per sé è più uno strumento teorico che serve a modelizzare fenomeni diversi (transistor, amplificatori, ecc...).

\par\medskip
\noindent
\textbf{\textsf{Collegamenti in parallelo}} \\
Per sommare i contributi alla corrente di più generatori di corrente, li disponiamo in parallelo:

\begin{center}
\begin{circuitikz}
    \draw (2,0) 
				to[short] (0,0) node[left] {$A$};
    
    \draw (2,-2) 
				to[short] (0,-2) node[left] {$B$};
		
		\draw (2,0 )to[european current source, l=$I_1$] (2,-2);
		\draw (4,0 )to[european current source, l=$I_2$] (4,-2);
		
		
    \draw (4,0) node[anchor=west,xshift=0.7cm] {\dots} (6,0);
    \draw (4,-2) node[anchor=west,xshift=0.7cm] {\dots} (6,-2); 
		
		\draw (8,0 )to[european current source, l=$I_n$] (8,-2);
        
    % Draw horizontal lines at the top and bottom
    \draw (0,0) -- (4,0);
    \draw (0,-2) -- (4,-2);

    \draw (6,0) -- (8,0);
    \draw (6,-2) -- (8,-2);

    % Bracket below the resistors
    \draw[decorate,decoration={brace,amplitude=10pt,mirror}] (0,-3) -- (8,-3)
        node[midway,below=10pt]{$n$ generatori};
\end{circuitikz}
\end{center}

Abbiamo che il contributo totale dei generatori equivale a quello di un singolo generatore $E_T$ di corrente:
$$ I_T = I_1 + I_2 + ... + I_n $$

\subsubsection{Resistenza interna}
Possiamo combinare i componenti visti finora per creare modelli più realistici.
Innanzitutto, è improbabile che un generatore reali applichi resistenza nulla alle cariche che vi scorrono dentro.
Aggiungiamo quindi una resistenza (solitamente piccola per i generatori di tensione ed elevata per i generatori di corrente) al generatore, che chiameremo \textbf{resistenza interna}.
Questa resistenza rappresenterà la potenza che viene dissipata per effetto Joule.

La resistenza si disporrà come segue per i diversi tipi di generatore:
\begin{itemize}
	\item \textbf{Generatore di tensione:} resistenza in serie;
\begin{center}
\begin{circuitikz}
    \draw (0,0) node[left] {$A$} 
        to[european voltage source, l=$V$] (2,0) 
				to[R, l=$R$] (4,0) node[right] {$B$};
\end{circuitikz}
\end{center}
	\item \textbf{Generatore di corrente:} resistenza in parallelo.
\begin{center}
\begin{circuitikz}
    \draw (2,0) 
				to[short] (0,0) node[left] {$A$};
    
    \draw (2,-2) 
				to[short] (0,-2) node[left] {$B$};
		
		\draw (2,0 )to[european current source, l=$I$] (2,-2);
		\draw (4,0 )to[R, l=$R$] (4,-2);
		
        
    % Draw horizontal lines at the top and bottom
    \draw (0,0) -- (4,0);
    \draw (0,-2) -- (4,-2);
\end{circuitikz}
\end{center}
\end{itemize}

Notiamo che i casi visti prima come impossibili, di generatori di tensione in parallelo e di generatori di corrente in serie, sono rappresentabili quando si rilascia l'ipotesi che i generatori siano ideali e si introducono resistenze interne.

\subsubsection{Generatori dipendenti}
I generatori dipendenti, detti anche controllati o pilotati, sono particolari tipi di generatore il cui voltaggio (o corrente) dipende dal valore del voltaggio (o corrente) di un'altro punto del circuito, scalato di un qualche coefficiente. 
Si indicano come i generatori indipendenti ma all'interno di un rombo invece che di un cerchio.

Abbiamo quindi 4 tipi fondamentali di generatori dipendenti:
\begin{itemize}
	\item \textbf{Generatori di tensione,} si indicano come:
\begin{center}
\begin{circuitikz}
    \draw (0,0) 
        to[european controlled voltage source] (2,0);
\end{circuitikz}
\end{center}
		\begin{itemize}
			\item \textbf{Generatore di tensione pilotato in tensione:} comandato dalla funzione:
				$$
				v(t) = \alpha \cdot v(t)
				$$
				su un punto arbitrario dove si calcola $i(t)$.
			\item \textbf{Generatore di tensione pilotato in corrente:} comandato dalla funzione:
				$$
				v(t) = \alpha \cdot i(t)
				$$
				su un punto arbitrario dove si calcola $v(t)$.
		\end{itemize}
	\item \textbf{Generatori di corrente,} si indicano come:
\begin{center}
\begin{circuitikz}
    \draw (0,0) 
        to[european controlled current source] (2,0);
\end{circuitikz}
\end{center}
		\begin{itemize}
			\item \textbf{Generatore di corrente pilotato in tensione:} comandato dalla funzione:
				$$
				i(t) = \alpha \cdot v(t)
				$$
				su un punto arbitrario dove si calcola $v(t)$.
			\item \textbf{Generatore di corrente pilotato in corrente:} comandato dalla funzione:
				$$
				i(t) = \alpha \cdot i(t)
				$$
				su un punto arbitrario dove si calcola $i(t)$.
		\end{itemize}
\end{itemize}

\par\smallskip 

Bisogna notare che, come già riportato, un generatore dipendente non è diverso da un dipolo passivo in termini di potenza: non porta nessuna energia esterna all'interno del circuito.
Si può anzi dire che è necessario avere almeno un generatore indipendente per avere spostamento di carica all'interno del circuito.

\subsection{Partitore di tensione}
Analizziamo il seguente circuito:

\begin{center}
\begin{circuitikz} 
	\draw
  (0,2) to[V, v=$e(t)$] (0,12) % Voltage source
  -- (2,12) 
  to[R, l=$R_1$] (2,10) 
  to[R, l=$R_2$] (2,8);

	\draw
	(2,7) to[R, l=$R_J$] (2,5);

	\draw
	(2,4) to[R, l=$R_n$] (2,2)
	-- (0, 2);

\draw (2,7.5) node[xshift=0.1] {\dots} (2,5);
	\draw (2,4.5) node[xshift=0.1] {\dots} (2,4); 
\end{circuitikz}
\end{center}

Reti di questo tipo prendono il nome di \textbf{partitori di tensione}, e hanno lo scopo di partizionare una certa differenza di potenziale in diverse frazioni proprie.

Poniamo di voler calcolare la caduta di potenziale su una particolare resistenza, diciamo la $R_J$. Avremo allora, dalla seconda legge di Kirchoff:
$$
-e(t) + R_1(t) i(t)+ R_2(t) i(t) + ... + R_J(t) i(t) + ... + R_n(t) i(t) = 0
$$
che raccogliendo la corrente comune diventa:
$$
e(t) = (R_1 + R_2 + ... + R_J + R_n) i(t) = i(t) \sum_{i=1}^n R_i
$$
somma delle resistenze per corrente.
A questo punto possiamo applicare la legge di Ohm per ottenere la caduta di potenziale:
$$
V_J(t) = R_J i(t) = e(t)\frac{R_j}{\sum_{i=1}^n R_i}
$$
cioè il rapporto fra la resistenza interessata e la resistenza complessiva del circuito, moltiplicata per la tensione.

\subsection{Partitore di corrente}
Analizziamo quindi il seguente circuito:

\begin{center}
\begin{circuitikz}
    \draw (2,-2) 
				-- (0,-2) 
				to[V, v=$e(t)$] (0, 0); % Voltage source
		
		\draw (2,0 )to[R, l=$R_1$] (2,-2);
		\draw (4,0 )to[R, l=$R_2$] (4,-2);
		
		
    \draw (4,0) node[anchor=west,xshift=0.7cm] {\dots} (6,0);
    \draw (4,-2) node[anchor=west,xshift=0.7cm] {\dots} (6,-2); 
		
		\draw (8,0 )to[R, l=$R_J$] (8,-2);
       
    \draw (8,0) node[anchor=west,xshift=0.7cm] {\dots} (10,0);
    \draw (8,-2) node[anchor=west,xshift=0.7cm] {\dots} (10,-2); 
		
		\draw (12,0 )to[R, l=$R_n$] (12,-2);

    % Draw horizontal lines at the top and bottom
    \draw (0,0) -- (4,0);
    \draw (0,-2) -- (4,-2);

    \draw (6,0) -- (8,0);
    \draw (6,-2) -- (8,-2);

    \draw (10,0) -- (12,0);
    \draw (10,-2) -- (12,-2);

\end{circuitikz}
\end{center}

Reti di questo tipo hanno uno scopo simile a quello della rete vista prima, solo riguardo alla corrente: prendono infatti il nome di \textbf{partitori di corrente}.

Poniamo di voler calcolare la corrente su una singola resistenza. 
Potremo dire che la corrente complessiva è, dalla prima legge di Kirchoff:

$$
I_T(t) = I_1(t) + I_2(t) + ... + I_J(t) + ... + I_n(t)
$$

Un'altro modo di ottenere queste correnti è dalla legge di Ohm, usando le conduttanze invece delle resistenze:
$$ 
I = \frac{V}{R} \Rightarrow I = GR, \quad I(t) = v(t) \sum_{i=1}^n G_i
$$
A questo punto, possiamo dire che la corrente nella J-esima resistenza vale:
$$
I_J(t) = v(t) G_n = I(t) \frac{G_J}{\sum_{i=1}^n} 
$$
cioè il rapporto fra la conduttanza (della resistenza) interessata e la conduttanza complessiva del circuito, moltiplicata per la corrente.

\end{document}


\documentclass[a4paper,11pt]{article}
\usepackage[a4paper, margin=8em]{geometry}

% usa i pacchetti per la scrittura in italiano
\usepackage[french,italian]{babel}
\usepackage[T1]{fontenc}
\usepackage[utf8]{inputenc}
\frenchspacing 

% usa i pacchetti per la formattazione matematica
\usepackage{amsmath, amssymb, amsthm, amsfonts}

% usa altri pacchetti
\usepackage{gensymb}
\usepackage{hyperref}
\usepackage{standalone}

% imposta il titolo
\title{Appunti Elettrotecnica}
\author{Luca Seggiani}
\date{2024}

% imposta lo stile
% usa helvetica
\usepackage[scaled]{helvet}
% usa palatino
\usepackage{palatino}
% usa un font monospazio guardabile
\usepackage{lmodern}

\renewcommand{\rmdefault}{ppl}
\renewcommand{\sfdefault}{phv}
\renewcommand{\ttdefault}{lmtt}

% disponi il titolo
\makeatletter
\renewcommand{\maketitle} {
	\begin{center} 
		\begin{minipage}[t]{.8\textwidth}
			\textsf{\huge\bfseries \@title} 
		\end{minipage}%
		\begin{minipage}[t]{.2\textwidth}
			\raggedleft \vspace{-1.65em}
			\textsf{\small \@author} \vfill
			\textsf{\small \@date}
		\end{minipage}
		\par
	\end{center}

	\thispagestyle{empty}
	\pagestyle{fancy}
}
\makeatother

% disponi teoremi
\usepackage{tcolorbox}
\newtcolorbox[auto counter, number within=section]{theorem}[2][]{%
	colback=blue!10, 
	colframe=blue!40!black, 
	sharp corners=northwest,
	fonttitle=\sffamily\bfseries, 
	title=~\thetcbcounter: #2, 
	#1
}

% disponi definizioni
\newtcolorbox[auto counter, number within=section]{definition}[2][]{%
	colback=red!10,
	colframe=red!40!black,
	sharp corners=northwest,
	fonttitle=\sffamily\bfseries,
	title=~\thetcbcounter: #2,
	#1
}

% U.D.M
\newcommand{\amp}{\ensuremath{\mathrm{A}}}
\newcommand{\volt}{\ensuremath{\mathrm{V}}}
\newcommand{\meter}{\ensuremath{\mathrm{m}}}
\newcommand{\second}{\ensuremath{\mathrm{s}}}
\newcommand{\farad}{\ensuremath{\mathrm{F}}}
\newcommand{\henry}{\ensuremath{\mathrm{H}}}
\newcommand{\siemens}{\ensuremath{\mathrm{S}}}

% circuiti
\usepackage{circuitikz}
\usetikzlibrary{babel}

% disegni
\usepackage{pgfplots}
\pgfplotsset{width=10cm,compat=1.9}

% disponi codice
\usepackage{listings}
\usepackage[table]{xcolor}

\lstdefinestyle{codestyle}{
		backgroundcolor=\color{black!5}, 
		commentstyle=\color{codegreen},
		keywordstyle=\bfseries\color{magenta},
		numberstyle=\sffamily\tiny\color{black!60},
		stringstyle=\color{green!50!black},
		basicstyle=\ttfamily\footnotesize,
		breakatwhitespace=false,         
		breaklines=true,                 
		captionpos=b,                    
		keepspaces=true,                 
		numbers=left,                    
		numbersep=5pt,                  
		showspaces=false,                
		showstringspaces=false,
		showtabs=false,                  
		tabsize=2
}

\lstdefinestyle{shellstyle}{
		backgroundcolor=\color{black!5}, 
		basicstyle=\ttfamily\footnotesize\color{black}, 
		commentstyle=\color{black}, 
		keywordstyle=\color{black},
		numberstyle=\color{black!5},
		stringstyle=\color{black}, 
		showspaces=false,
		showstringspaces=false, 
		showtabs=false, 
		tabsize=2, 
		numbers=none, 
		breaklines=true
}

\lstdefinelanguage{javascript}{
	keywords={typeof, new, true, false, catch, function, return, null, catch, switch, var, if, in, while, do, else, case, break},
	keywordstyle=\color{blue}\bfseries,
	ndkeywords={class, export, boolean, throw, implements, import, this},
	ndkeywordstyle=\color{darkgray}\bfseries,
	identifierstyle=\color{black},
	sensitive=false,
	comment=[l]{//},
	morecomment=[s]{/*}{*/},
	commentstyle=\color{purple}\ttfamily,
	stringstyle=\color{red}\ttfamily,
	morestring=[b]',
	morestring=[b]"
}

% disponi sezioni
\usepackage{titlesec}

\titleformat{\section}
	{\sffamily\Large\bfseries} 
	{\thesection}{1em}{} 
\titleformat{\subsection}
	{\sffamily\large\bfseries}   
	{\thesubsection}{1em}{} 
\titleformat{\subsubsection}
	{\sffamily\normalsize\bfseries} 
	{\thesubsubsection}{1em}{}

% disponi alberi
\usepackage{forest}

\forestset{
	rectstyle/.style={
		for tree={rectangle,draw,font=\large\sffamily}
	},
	roundstyle/.style={
		for tree={circle,draw,font=\large}
	}
}

% disponi algoritmi
\usepackage{algorithm}
\usepackage{algorithmic}
\makeatletter
\renewcommand{\ALG@name}{Algoritmo}
\makeatother

% disponi numeri di pagina
\usepackage{fancyhdr}
\fancyhf{} 
\fancyfoot[L]{\sffamily{\thepage}}

\makeatletter
\fancyhead[L]{\raisebox{1ex}[0pt][0pt]{\sffamily{\@title \ \@date}}} 
\fancyhead[R]{\raisebox{1ex}[0pt][0pt]{\sffamily{\@author}}}
\makeatother

\begin{document}
% sezione (data)
\section{Lezione del 03-10-24}

% stili pagina
\thispagestyle{empty}
\pagestyle{fancy}

% testo
\subsection{Principio di sovrapposizione degli effetti}
Possiamo sfruttare la linearità delle equazioni che descrivono i circuiti visti finora per dimostrare il seguente risultato:
\begin{theorem}{Principio di sovrapposizione}
	La risposta di una rete alla sollecitazione di più generatori indipendenti può essere ottenuta considerando ciascun generatore \textit{separatamente attivo} e sommando algebricamente le risposte.
\end{theorem}

Nel dettaglio, in un circuito con generatori di voltaggio e tensione, questo significa prendere ogni generatore, uno per volta, e considerarlo come l'unico attivo, disattivando gli altri.
Disattivare un generatore significa impostare il voltaggio (o la corrente) emessa a 0. Quindi:
\begin{itemize}
	\item \textbf{Generatori di tensione:} disattivare significa trasformare in cortocircuiti ($e(t) = 0$);
	\item \textbf{Generatori di corrente:} disattivare significa trasformare in circuiti aperti ($I(t) = 0$).
\end{itemize}

Algebricamente, se vogliamo ricavare la corrente o il voltaggio su una maglia di un circuito con $n$ generatori, possiamo porla come la somma delle $I'_x + I''_x + ... + I_x^n$ correnti ($V'_x + V''_x + ... + V_x^n$ voltaggi) ricavati da ogni generatore preso singolarmente.

\par\smallskip

Facciamo un'esempio:

\begin{center}
\begin{circuitikz}
    \draw (2,-2) 
				-- (0,-2) 
				to[ european voltage source, v=$e(t)$] (0, 0)
				to[R, l=$R$] (0,2);	
		
		\draw (2,2 )to[ european current source, i=$J(t)$] (2,-2);
		\draw (4,2 )to[R, l=$R$] (4,-2);
		
    % Draw horizontal lines at the top and bottom
    \draw (0,2) -- (4,2);
    \draw (0,-2) -- (4,-2);

\end{circuitikz}
\end{center}

Poniamo di voler calcolare la potenza dissipata dal resistore destro.
Potremo applicare quanto ricavato da Ohm, ergo:
$$
P_R = R I_x^2
$$
con $I_x$ la corrente che passa sul resistore destro.

Questa corrente si calcola agilmente attraverso il principio di sovrapposizione.
Poniamo quindi $I_x = I_x' + I_x''$, con $I_x'$ data dal generatore di tensione $e(t)$, e $I_x''$ data dal generatore di corrente $J(t)$.

Calcoliamo quindi le correnti trovate in entrambi i casi:
\begin{itemize}
	\item $I_x'$, \textbf{generatore di tensione attivo.} 
		Si ha che il generatore di corrente, disattivato, equivale al circuito aperto, ergo l'unica maglia è quella che collega in serie le due resistenze.
		Da qui si calcola velocemente che:
		$$ I_x' = \frac{e(t)}{2R} $$
		oppure si applica la seconda legge di Kirchoff:
		$$ e(t) - I_x' R - I_x'R = 0 $$
		da cui si ricava lo stesso risultato.
	\item $I_x''$, \textbf{generatore di corrente attivo.}
		Si ha che il generatore di tensione, disattivato, equivale al circuito chiuso, ergo abbiamo quello che è effettivamente un partitore di corrente formato da due resistenze in parallelo.
		Visto che le due resistenze sono uguali, possimo dire che la corrente è partizionata ugualmente, facendo attenzione ai segni (il generatore di corrente ha direzione opposta al generatore di tensione), cioè:
		$$ I_x'' = -J(t) \frac{G_R}{G_{tot.}} = -J(t) \frac{R}{R + R} = -\frac{J(t)}{2}$$
\end{itemize}

A questo punto otteniamo la corrente totale come:
$$ I_x = I_x' + I_x'' = \frac{e(t)}{2R} - \frac{J(t)}{2} $$
e la potenza come:
$$ P_R = R \left( \frac{e(t)}{2R} - \frac{J(t)}{2} \right)^2 $$



\end{document}


\documentclass[a4paper,11pt]{article}
\usepackage[a4paper, margin=8em]{geometry}

% usa i pacchetti per la scrittura in italiano
\usepackage[french,italian]{babel}
\usepackage[T1]{fontenc}
\usepackage[utf8]{inputenc}
\frenchspacing 

% usa i pacchetti per la formattazione matematica
\usepackage{amsmath, amssymb, amsthm, amsfonts}

% usa altri pacchetti
\usepackage{gensymb}
\usepackage{hyperref}
\usepackage{standalone}

% imposta il titolo
\title{Appunti Elettrotecnica}
\author{Luca Seggiani}
\date{2024}

% imposta lo stile
% usa helvetica
\usepackage[scaled]{helvet}
% usa palatino
\usepackage{palatino}
% usa un font monospazio guardabile
\usepackage{lmodern}

\renewcommand{\rmdefault}{ppl}
\renewcommand{\sfdefault}{phv}
\renewcommand{\ttdefault}{lmtt}

% disponi il titolo
\makeatletter
\renewcommand{\maketitle} {
	\begin{center} 
		\begin{minipage}[t]{.8\textwidth}
			\textsf{\huge\bfseries \@title} 
		\end{minipage}%
		\begin{minipage}[t]{.2\textwidth}
			\raggedleft \vspace{-1.65em}
			\textsf{\small \@author} \vfill
			\textsf{\small \@date}
		\end{minipage}
		\par
	\end{center}

	\thispagestyle{empty}
	\pagestyle{fancy}
}
\makeatother

% disponi teoremi
\usepackage{tcolorbox}
\newtcolorbox[auto counter, number within=section]{theorem}[2][]{%
	colback=blue!10, 
	colframe=blue!40!black, 
	sharp corners=northwest,
	fonttitle=\sffamily\bfseries, 
	title=~\thetcbcounter: #2, 
	#1
}

% disponi definizioni
\newtcolorbox[auto counter, number within=section]{definition}[2][]{%
	colback=red!10,
	colframe=red!40!black,
	sharp corners=northwest,
	fonttitle=\sffamily\bfseries,
	title=~\thetcbcounter: #2,
	#1
}

% U.D.M
\newcommand{\amp}{\ensuremath{\mathrm{A}}}
\newcommand{\volt}{\ensuremath{\mathrm{V}}}
\newcommand{\meter}{\ensuremath{\mathrm{m}}}
\newcommand{\second}{\ensuremath{\mathrm{s}}}
\newcommand{\farad}{\ensuremath{\mathrm{F}}}
\newcommand{\henry}{\ensuremath{\mathrm{H}}}
\newcommand{\siemens}{\ensuremath{\mathrm{S}}}

% circuiti
\usepackage{circuitikz}
\usetikzlibrary{babel}

% disegni
\usepackage{pgfplots}
\pgfplotsset{width=10cm,compat=1.9}

% disponi codice
\usepackage{listings}
\usepackage[table]{xcolor}

\lstdefinestyle{codestyle}{
		backgroundcolor=\color{black!5}, 
		commentstyle=\color{codegreen},
		keywordstyle=\bfseries\color{magenta},
		numberstyle=\sffamily\tiny\color{black!60},
		stringstyle=\color{green!50!black},
		basicstyle=\ttfamily\footnotesize,
		breakatwhitespace=false,         
		breaklines=true,                 
		captionpos=b,                    
		keepspaces=true,                 
		numbers=left,                    
		numbersep=5pt,                  
		showspaces=false,                
		showstringspaces=false,
		showtabs=false,                  
		tabsize=2
}

\lstdefinestyle{shellstyle}{
		backgroundcolor=\color{black!5}, 
		basicstyle=\ttfamily\footnotesize\color{black}, 
		commentstyle=\color{black}, 
		keywordstyle=\color{black},
		numberstyle=\color{black!5},
		stringstyle=\color{black}, 
		showspaces=false,
		showstringspaces=false, 
		showtabs=false, 
		tabsize=2, 
		numbers=none, 
		breaklines=true
}

\lstdefinelanguage{javascript}{
	keywords={typeof, new, true, false, catch, function, return, null, catch, switch, var, if, in, while, do, else, case, break},
	keywordstyle=\color{blue}\bfseries,
	ndkeywords={class, export, boolean, throw, implements, import, this},
	ndkeywordstyle=\color{darkgray}\bfseries,
	identifierstyle=\color{black},
	sensitive=false,
	comment=[l]{//},
	morecomment=[s]{/*}{*/},
	commentstyle=\color{purple}\ttfamily,
	stringstyle=\color{red}\ttfamily,
	morestring=[b]',
	morestring=[b]"
}

% disponi sezioni
\usepackage{titlesec}

\titleformat{\section}
	{\sffamily\Large\bfseries} 
	{\thesection}{1em}{} 
\titleformat{\subsection}
	{\sffamily\large\bfseries}   
	{\thesubsection}{1em}{} 
\titleformat{\subsubsection}
	{\sffamily\normalsize\bfseries} 
	{\thesubsubsection}{1em}{}

% disponi alberi
\usepackage{forest}

\forestset{
	rectstyle/.style={
		for tree={rectangle,draw,font=\large\sffamily}
	},
	roundstyle/.style={
		for tree={circle,draw,font=\large}
	}
}

% disponi algoritmi
\usepackage{algorithm}
\usepackage{algorithmic}
\makeatletter
\renewcommand{\ALG@name}{Algoritmo}
\makeatother

% disponi numeri di pagina
\usepackage{fancyhdr}
\fancyhf{} 
\fancyfoot[L]{\sffamily{\thepage}}

\makeatletter
\fancyhead[L]{\raisebox{1ex}[0pt][0pt]{\sffamily{\@title \ \@date}}} 
\fancyhead[R]{\raisebox{1ex}[0pt][0pt]{\sffamily{\@author}}}
\makeatother

\begin{document}
% sezione (data)
\section{Lezione del 04-10-24}

% stili pagina
\thispagestyle{empty}
\pagestyle{fancy}

% testo
\subsection{Metodo del tableau}
Vediamo un metodo generale per la risoluzione completa di un circuito, detto metodo delle correnti di ramo, o \textit{del tablau}.
Di base, si hanno $r$ incognite per $r$ rami, ergo le correnti che passano nei rispettivi rami.
Applichiamo quindi il seguente algoritmo:

\begin{enumerate}
	\item Dare un nome a tutti i nodi ($A_1, A_2, ..., A_n$);
	\item Dare un nome a tutte le correnti di ramo ($I_1, I_2, ..., I_n$).
		Si noti che in questo passaggio il verso delle correnti è puramente arbitrario: il segno positivo o negativo della corrente trovata ci darà il verso rispetto al riferimento (associato o meno) scelto.
	\item Scrivere $n - 1$ equazioni per $n - 1$ nodi applicando la prima legge di Kirchoff.
		Si scelgono $n - 1$ equazioni perché, essendo la corrente vincolata dalla legge di Kirchoff, la $n$-esima equazione sarà linearmente dipendente alle altre, ergo ridondante in informazioni riguardo al circuito.
	\item Scrivere $r - n + 1$ equazioni con la seconda legge di Kirchoff.
		La prima maglia si sceglie a caso, mentre le successive maglie si scelgono cancellando un ramo della maglia scelta al passaggio scorso.
		Questa procedura è, ancora una volta, necessaria per evitare la dipendenza lineare delle equazioni trovate.
\end{enumerate}

Seguendo questi passaggi, si trova un sistema lineare in $r$ variabili con $r$ equazioni, che sappiamo essere determinato e risolvibile come:
$$ Ax = b \Rightarrow x = A^{-1} b $$

\par\smallskip

Prendiamo in esempio il seguente circuito:

\begin{center}
\begin{circuitikz}
	\draw (0,0)
		-- (0,1.5)
		to[R, l=$R_1$] (5,1.5)
		-- (5, 0);
	\draw (0,0)
		to[R, l=$R_2$] (2.5, 0)
		to[R, l=$R_3$] (5, 0);
	\draw (0,0)
		to[R, l=$R_4$] (0, -3);
	\draw (2.5,0)
		to[R, l=$R_5$] (2.5, -3);
	\draw (5,-3)
		to[ european voltage source, v=$E_2$] (5, 0);
	\draw (5, -3)
		to[R, l=$R_6$] (2.5, -3)
		to[ european voltage source, v=$E_1$] (0, -3);
\end{circuitikz}
\end{center}

Eseguiamo i passi in ordine.
\begin{enumerate}
	\item Innanzitutto, si danno i nomi $A$, $B$, $C$ e $D$ ai nodi del circuito:

\begin{center}
\begin{circuitikz}
	\draw (0,0)
		-- (0,1.5)
		to[R, l=$R_1$] (5,1.5)
		-- (5, 0);
	\draw (0,0)
		to[R, l=$R_2$] (2.5, 0)
		to[R, l=$R_3$] (5, 0);
	\draw (0,0)
		to[R, l=$R_4$] (0, -3);
	\draw (2.5,0)
		to[R, l=$R_5$] (2.5, -3);
	\draw (5,-3)
		to[ european voltage source, v=$E_2$] (5, 0);
	\draw (5, -3)
		to[R, l=$R_6$] (2.5, -3)
		to[ european voltage source, v=$E_1$] (0, -3);

	\draw (0,0) node[circ] {};
	\draw (0,0) node[left] {A};
	
	\draw (2.5,0) node[circ] {};
	\draw (2.5,0) node[above] {B};

	\draw (5,0) node[circ] {};
	\draw (5,0) node[right] {C};

	\draw (2.5,-3) node[circ] {};
	\draw (2.5,-3) node[below] {D};
\end{circuitikz}
\end{center}

	\item Quindi, si danno i nomi $I_1$, $I_2$, $I_3$, $I_4$, $I_5$ e $I_6$, e i versi di percorrenza alle correnti sui rispettivi rami:

\begin{center}
\begin{circuitikz}
	\draw (0,0)
		-- (0,1.5)
		to[R, l=$R_1$, i=$I_1$] (5,1.5)
		-- (5, 0);
	\draw (0,0)
		to[R, l=$R_2$, i_ = $I_2$] (2.5, 0);
		
	\draw (5, 0) to[R, l_=$R_3$, i = $I_3$] (2.5, 0);
	
	\draw (0,0)
		to[R, l=$R_4$, i = $I_4$] (0, -3);
	\draw (2.5,0)
		to[R, l=$R_5$, i = $I_5$] (2.5, -3);
	\draw (5,-3)
		to[ european voltage source, v=$E_2$] (5, 0);
	\draw (2., -3)
		to[R, l_=$R_6$, i = $I_6$] (5, -3);
	\draw (2.5, -3)
		to[ european voltage source, v=$E_1$] (0, -3);

	\draw (0,0) node[circ] {};
	\draw (0,0) node[left] {A};
	
	\draw (2.5,0) node[circ] {};
	\draw (2.5,0) node[above] {B};

	\draw (5,0) node[circ] {};
	\draw (5,0) node[right] {C};

	\draw (2.5,-3) node[circ] {};
	\draw (2.5,-3) node[below] {D};
\end{circuitikz}
\end{center}

\item A questo punto possiamo applicare la prima legge di Kirchoff su $n-1$ nodi, diciamo $A$, $B$ e $C$:
$$
A: I_1 + I_2 + I_4 = 0 
$$
$$
B: I_2 + I_3 = I_5 
$$
$$
C: I_1 + I_6 = I_3
$$
La legge di Kirchoff applicata al nodo $D$ qui sarebbe inutile, in quanto fornirebbe:
$$
D: I_5 + I_4 = I_6
$$
che potevamo già ottenere sostituendo la $C$ alla $B$:
$$
C: I_2 + I_3 = I_5 \Rightarrow I_2 + I_1 + I_6 = I_5  
$$
e a questo punto usando la $A$ come $I_1 + I_2 = -I_4$:
$$
\Rightarrow -I_4 + I_6 = I_5 \Rightarrow I_5 + I_4 = I_6
$$

\item Completiamo il sistema introducendo $r - n + 1 = 3$ equazioni ricavate applicando la seconda legge di Kirchoff in senso antiorario.
	Dovremo prendere i segni delle cadute di potenziale tenendo conto della direzione della corrente scelta su quel ramo.
	
	Prendiamo innanzitutto la maglia in basso a sinistra:
	$$ -E_1 + R_5 I_5 + R_2 I _2 - R_4 I_4 = 0 $$
	ed eliminiamo il ramo di corrente $I_4$.
	Quindi prendiamo la maglia formata dalla parte superiore e dalla parte in basso a destra del circuito:
	$$ E_2 + R_1 I_1 - R_2 I_2 - R_5 I_5 - R_6 I_6 = 0 $$
	ed eliminiamo il ramo di corrente $I_5$ o $I_6$ (è irrilevante per il prossimo passaggio), in quanto prenderemo come ultima la maglia in alto:
	$$ R_1 I_1 - R_2 I_2 + R_3 I_3 = 0 $$
\end{enumerate}

Conclusi questi passaggi, abbiamo ricavato il sistema lineare:
\[
	\begin{cases}
I_1 + I_2 + I_4 = 0 \\ 
I_2 + I_3 = I_5 \\
I_1 + I_6 = I_3 \\
-E_1 + R_5 I_5 + R_2 I _2 - R_4 I_4 = 0 \\
E_2 + R_1 I_1 - R_2 I_2 - R_5 I_5 - R_6 I_6 = 0 \\
R_1 I_1 - R_2 I_2 + R_3 I_3 = 0
	\end{cases}
\]

Possiamo ricavare le matrici $A$, $x$ e $b$ del sistema, ed esprimerlo come $Ax = b$:
\[
\underbrace{\begin{pmatrix}
		1 & 1 & 0 & 1 & 0 & 0 \\ 
		0 & 1 & 1 & 0 & -1 & 0 \\ 
		1 & 0 & -1 & 0 & 0 & 1 \\ 
		0 & R_2 & 0 & -R_4 & R_5 & 0 \\ 
		R_1 & -R_2 & 0 & 0 & -R_5 & -R_6 \\ 
		R_1 & -R_2 & R_3 & 0 & 0 & 0 \\ 
\end{pmatrix}}_{\mathbf{A}} 
\underbrace{\begin{pmatrix}
		I_1 \\
		I_2 \\
		I_3 \\
		I_4 \\
		I_5 \\
		I_6 \\
\end{pmatrix}}_{\mathbf{x}} 
= 
\underbrace{\begin{pmatrix}
		0 \\ 
		0 \\ 
		0 \\ 
		E_1 \\ 
		-E_2 \\ 
		0
\end{pmatrix}}_{\mathbf{b}}
\]

Il sistema si risolve con qualsiasi metodo di risoluzione di sistemi lineari.

\par\smallskip

Notiamo come il metodo del tableu ci permette di notare due fatti già visti sui circuiti.

Innanzitutto, abbiamo che fra i termini noti $b$ compaiono solo le correnti e i voltaggi dei generatori indipendenti.
Senza di questi, il sistema sarebbe di equazioni omogenee, e quindi ammetterebbe unica soluzione $0$.
Questa affermazione è equivalente a quella già vista: sono i generatori indipendenti a portare energia dentro il circuito.

Possiamo poi dire che il il metodo del tableau è compatibile col principio di sovrapposizione.
Presi separatamente i generatori, infatti, avremo $x', x'', ..., x^n$ soluzioni date da $b', b'', ..., b^n$ vettori $b$, ergo:
$$ x = x' + x'' + ... + x^n = A^{-1} b' + A^{-1} b'' + ... + A^{-1} b^n$$

Intuitivamente, la matrice $A$ rappresenta la risposta del circuito a diversi stimoli $b', b'', ..., b^n$.
Nell'esempio, questi sono i generatori indipendenti:

$$
b':
\begin{pmatrix}
		0 \\ 
		0 \\ 
		0 \\ 
		E_1 \\ 
		0 \\ 
		0
\end{pmatrix}, \quad 
b'':
\begin{pmatrix}
		0 \\ 
		0 \\ 
		0 \\ 
		0 \\ 
		-E_2 \\ 
		0
\end{pmatrix}
$$

\subsubsection{Circuiti con generatori di corrente}
Quando si applica il metodo del tableau ad un circuito con generatori di corrente, bisogna notare che questi riducono il numero di incognite (le correnti) del sistema, cioè portano il numero di correnti di ramo a $r - N_I$ dove $N_I$ è il numero di generatori di corrente.
In seguito, bisogna fare attenzione a tagliare preliminarmente i rami contenenti generatori di correnti prima di eseguire il punto 4) dell'algoritmo, visto che il potenziale su quei rami potrebbe essere qualsiasi. 

\par\smallskip 

Prendiamo in esempio il seguente circuito:

\begin{center}
\begin{circuitikz}
	\draw (0,0)
		-- (0,1.5)
		to[R, l=$R_3$] (5,1.5)
		-- (5, 0);
	\draw (0,0)
		to[ european current source, i = $I$] (2.5, 0)
		to[short] (5, 0);
	\draw (0,0)
		to[R, l=$R_1$] (0, -3);
	\draw (2.5,0)
		to[R, l=$R_2$] (2.5, -3);
	\draw (5,-3)
		to[ european voltage source, v=$E$] (5, 0);
	\draw (5, -3)
		to[short] (2.5, -3)
		to[short] (0, -3);
\end{circuitikz}
\end{center}

Si ricava che i nodi e le correnti sono le seguenti, facendo attenzione alla disposizione di $B$, che viene detto \textbf{macronodo}.
Un nodo del genere è utile in quanto il potenziale di ogni suo punto è effettivamente equivalente.

\begin{center}
\begin{circuitikz}
	\draw (0,0)
		-- (0,1.5)
		to[R, l=$R_3$, i = $I_3$] (5,1.5)
		-- (5, 0);
	\draw (0,0)
		to[ european current source, i = $I$] (2.5, 0)
		to[short] (5, 0);
	\draw (0,0)
		to[R, l=$R_1$, i = $I_1$] (0, -3);
	\draw (2.5,0)
		to[R, l=$R_2$, i = $I_2$] (2.5, -3);
	\draw (5,-3)
		to[ european voltage source, v=$E$, i = $I_4$] (5, 0);
	\draw (5, -3)
		to[short] (2.5, -3)
		to[short] (0, -3);

	\draw (0,0) node[circ] {};
	\draw (0,0) node[left] {A};
	
	\draw (2.5,0) node[circ] {};
	\draw (3.75,0) node[above] {B};

	\draw (5,0) node[circ] {};

	\draw[line width=1.25mm] (2.5, 0) -- (5,0);

	\draw (2.5,-3) node[circ] {};
	\draw (2.5,-3) node[below] {C};

\end{circuitikz}
\end{center}

Calcoliamo quindi la prima legge di Kirchoff sui nodi $A$ e $B$:
$$
A: I + I_1 + I_3 = 0
$$
$$
B: I + I_3 = I_2 + I_4
$$

Adesso calcoliamo la seconda legge di Kirchoff sulle maglie.
Abbiamo detto che dobbiamo ignorare i rami contenenti generatori di corrente.
Questo è vero perché provando, ad esempio, il ramo in basso a sinistra, avremmo:
$$
-V_I - R_1 I_1 + R_2 I_2 = 0
$$
dove $V_I$ è la differenza di potenziale sul generatore di corrente, che sappiamo può essere qualsiasi.

Usiamo quindi altre due maglie: quella data dalle maglie in alto e in basso a sinistra, e quella in basso a destra:
$$
R_3 I_3 - R_1 I_1 + R_2 I_2 = 0
$$
$$
-E + R_2 I_2 = 0
$$

Ricaviamo il sistema:
\[
	\begin{cases}
I + I_1 + I_3 = 0 \\ 
I + I_3 = I_2 + I_4 \\ 
R_3 I_3 - R_1 I_1 + R_2 I_2 = 0 \\
-E + R_2 I_2 = 0
	\end{cases}
\]

ovvero:

\[
\begin{pmatrix}
		1 & 0 & 1 & 0 \\ 
		0 & -1 & 1 & -1 \\ 
		-R_1 & R_2 & R_3 & 0 \\ 
		0 & R_2 & 0 & 0
\end{pmatrix} 
\begin{pmatrix}
		I_1 \\
		I_2 \\
		I_3 \\
		I_4 \\
\end{pmatrix} 
= 
\begin{pmatrix}
		-I \\ 
		-I \\ 
		0 \\ 
		E \\ 
\end{pmatrix}
\]

\subsubsection{Circuiti con generatori pilotati}
I generatori pilotati possono esprimere nuove incognite, in quanto la grandezza pilota è una corrente o una tensione ricavata su qualche ramo del circuito.
Inoltre, bisogna comunque tagliare fuori dal circuito o meno la sorgente in base al fatto che essa sia di voltaggio o di corrente, come avevamo già visto per i generatori indipendenti.

\par\smallskip

Prendiamo in esempio il seguente circuito:

\begin{center}
	\begin{circuitikz}[scale=1.2]
    \draw (2,-2) 
				-- (0,-2) 
				to[R, l=$R_1$] (0,1)
				-- (2, 1);	
		\draw (2,1)
			to [R, l = $R_2$] (4, 1)
			to [R, l = $R_3$] (6, 1)
			to [R, l = $R_4$] (8, 1)
			to [ european voltage source, v=$E_2$] (8, -2)
			-- (0, -2);

		\draw (2, -2)
			to [ european voltage source, v_=$E_1$] (2, 1);
		\draw (4, 1)
			to [ european current source, i=$J_1$] (4, -2);
		\draw (6, -2)
			to [ european controlled current source, i=$\alpha I_4$] (6, 1);
\end{circuitikz}
\end{center}

dove compare una sorgente pilotata in corrente di corrente.
Per quanto riguarda le maglie, la considereremo come una qualsiasi sorgente indipendente di corrente.
La grandezza pilota, invece, è una corrente ($I_4$) comune: fosse stata un voltaggio, avremmo dovuto esplicitarla con una variabile ed un equazione a sé (ad esempio $v_{CD} = R_4 I_4)$).

Disponiamo quindi nodi e cariche (si noti anche qui il macronodo $D$):

\begin{center}
	\begin{circuitikz}[scale=1.2]
    \draw (2,-2) 
				-- (0,-2) 
				to[R, l=$R_1$, i = $I_1$] (0,1)
				-- (2, 1);	
		\draw (2,1)
			to [R, l = $R_2$, i = $I_2$] (4, 1)
			to [R, l = $R_3$, i = $I_3$] (6, 1)
			to [R, l = $R_4$, i = $I_4$] (8, 1)
			to [ european voltage source, v=$E_2$] (8, -2)
			-- (0, -2);

		\draw (2, -2)
			to [ european voltage source, v_=$E_1$, i=$I_5$] (2, 1);
		\draw (4, 1)
			to [ european current source, i=$J_1$] (4, -2);
		\draw (6, -2)
			to [ european controlled current source, i=$\alpha I_4$] (6, 1);


		\draw (2,1) node[circ] {};
		\draw (2,1) node[above] {A};

		\draw (4,1) node[circ] {};
		\draw (4,1) node[above] {B};

		\draw (6,1) node[circ] {};
		\draw (6,1) node[above] {C};

		\draw (2,-2) node[circ] {};
		\draw (6,-2) node[circ] {};
		\draw[line width=1.25mm] (2, -2) -- (6,-2);

		\draw (4,-2) node[below] {D};
\end{circuitikz}
\end{center}

Calcoliamo con la prima legge di Kirchoff:
$$
A: I_1 = I_2 + I_5
$$
$$
B: I_2 = I_3 + J_1
$$
$$
D: I_5 + J_1 + I_4 = I_1 + \alpha I_4
$$
e con la seconda, applicata alla maglia sinistra e alle tre rimanenti (ricordando che i generatori di corrente vanno ignorati):
$$
E_1 + R_1 = 0
$$
$$
-E_1 - E_2 + R_4 I_4 + R_3 I_3 + R_2 I_2 = 0
$$
da cui il sistema:
\[
	\begin{cases}			
I_1 = I_2 + I_5 \\ 
I_2 = I_3 + J_1 \\ 
I_5 + J_1 + I_4 = I_1 + \alpha I_4 \\
E_1 + R_1 I_1 = 0 \\ 
-E_1 - E_2 + R_4 I_4 + R_3 I_3 + R_2 I_2 = 0 
	\end{cases}
\]

ovvero:

\[
\begin{pmatrix}
	1 & -2 & 0 & 0 & 1 \\ 
	0 & 1 & -1 & 0 & 0 \\ 
	-1 & 0 & 0 & 1 - \alpha & 1 \\ 
	R_1 & 0 & 0 & 0 & 0 \\
	0 & R_2 & R_3 & R_4 & 0
\end{pmatrix} 
\begin{pmatrix}
		I_1 \\
		I_2 \\
		I_3 \\
		I_4 \\
		I_5
\end{pmatrix} 
= 
\begin{pmatrix}
	0 \\ 
	J_1 \\ 
	-J_1 \\ 
	-E_1 \\ 
	E_1 + E_2
\end{pmatrix}
\]

\end{document}

\end{document}