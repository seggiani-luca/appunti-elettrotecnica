
\documentclass[a4paper,11pt]{article}
\usepackage[a4paper, margin=8em]{geometry}

% usa i pacchetti per la scrittura in italiano
\usepackage[french,italian]{babel}
\usepackage[T1]{fontenc}
\usepackage[utf8]{inputenc}
\frenchspacing 

% usa i pacchetti per la formattazione matematica
\usepackage{amsmath, amssymb, amsthm, amsfonts}

% usa altri pacchetti
\usepackage{gensymb}
\usepackage{hyperref}
\usepackage{standalone}

% imposta il titolo
\title{Appunti Elettrotecnica}
\author{Luca Seggiani}
\date{2024}

% imposta lo stile
% usa helvetica
\usepackage[scaled]{helvet}
% usa palatino
\usepackage{palatino}
% usa un font monospazio guardabile
\usepackage{lmodern}

\renewcommand{\rmdefault}{ppl}
\renewcommand{\sfdefault}{phv}
\renewcommand{\ttdefault}{lmtt}

% disponi il titolo
\makeatletter
\renewcommand{\maketitle} {
	\begin{center} 
		\begin{minipage}[t]{.8\textwidth}
			\textsf{\huge\bfseries \@title} 
		\end{minipage}%
		\begin{minipage}[t]{.2\textwidth}
			\raggedleft \vspace{-1.65em}
			\textsf{\small \@author} \vfill
			\textsf{\small \@date}
		\end{minipage}
		\par
	\end{center}

	\thispagestyle{empty}
	\pagestyle{fancy}
}
\makeatother

% disponi teoremi
\usepackage{tcolorbox}
\newtcolorbox[auto counter, number within=section]{theorem}[2][]{%
	colback=blue!10, 
	colframe=blue!40!black, 
	sharp corners=northwest,
	fonttitle=\sffamily\bfseries, 
	title=~\thetcbcounter: #2, 
	#1
}

% disponi definizioni
\newtcolorbox[auto counter, number within=section]{definition}[2][]{%
	colback=red!10,
	colframe=red!40!black,
	sharp corners=northwest,
	fonttitle=\sffamily\bfseries,
	title=~\thetcbcounter: #2,
	#1
}

% U.D.M
\newcommand{\amp}{\ensuremath{\mathrm{A}}}
\newcommand{\volt}{\ensuremath{\mathrm{V}}}
\newcommand{\meter}{\ensuremath{\mathrm{m}}}
\newcommand{\second}{\ensuremath{\mathrm{s}}}
\newcommand{\farad}{\ensuremath{\mathrm{F}}}
\newcommand{\henry}{\ensuremath{\mathrm{H}}}
\newcommand{\siemens}{\ensuremath{\mathrm{S}}}

% circuiti
\usepackage{circuitikz}
\usetikzlibrary{babel}

% disegni
\usepackage{pgfplots}
\pgfplotsset{width=10cm,compat=1.9}

% disponi codice
\usepackage{listings}
\usepackage[table]{xcolor}

\lstdefinestyle{codestyle}{
		backgroundcolor=\color{black!5}, 
		commentstyle=\color{codegreen},
		keywordstyle=\bfseries\color{magenta},
		numberstyle=\sffamily\tiny\color{black!60},
		stringstyle=\color{green!50!black},
		basicstyle=\ttfamily\footnotesize,
		breakatwhitespace=false,         
		breaklines=true,                 
		captionpos=b,                    
		keepspaces=true,                 
		numbers=left,                    
		numbersep=5pt,                  
		showspaces=false,                
		showstringspaces=false,
		showtabs=false,                  
		tabsize=2
}

\lstdefinestyle{shellstyle}{
		backgroundcolor=\color{black!5}, 
		basicstyle=\ttfamily\footnotesize\color{black}, 
		commentstyle=\color{black}, 
		keywordstyle=\color{black},
		numberstyle=\color{black!5},
		stringstyle=\color{black}, 
		showspaces=false,
		showstringspaces=false, 
		showtabs=false, 
		tabsize=2, 
		numbers=none, 
		breaklines=true
}

\lstdefinelanguage{javascript}{
	keywords={typeof, new, true, false, catch, function, return, null, catch, switch, var, if, in, while, do, else, case, break},
	keywordstyle=\color{blue}\bfseries,
	ndkeywords={class, export, boolean, throw, implements, import, this},
	ndkeywordstyle=\color{darkgray}\bfseries,
	identifierstyle=\color{black},
	sensitive=false,
	comment=[l]{//},
	morecomment=[s]{/*}{*/},
	commentstyle=\color{purple}\ttfamily,
	stringstyle=\color{red}\ttfamily,
	morestring=[b]',
	morestring=[b]"
}

% disponi sezioni
\usepackage{titlesec}

\titleformat{\section}
	{\sffamily\Large\bfseries} 
	{\thesection}{1em}{} 
\titleformat{\subsection}
	{\sffamily\large\bfseries}   
	{\thesubsection}{1em}{} 
\titleformat{\subsubsection}
	{\sffamily\normalsize\bfseries} 
	{\thesubsubsection}{1em}{}

% disponi alberi
\usepackage{forest}

\forestset{
	rectstyle/.style={
		for tree={rectangle,draw,font=\large\sffamily}
	},
	roundstyle/.style={
		for tree={circle,draw,font=\large}
	}
}

% disponi algoritmi
\usepackage{algorithm}
\usepackage{algorithmic}
\makeatletter
\renewcommand{\ALG@name}{Algoritmo}
\makeatother

% disponi numeri di pagina
\usepackage{fancyhdr}
\fancyhf{} 
\fancyfoot[L]{\sffamily{\thepage}}

\makeatletter
\fancyhead[L]{\raisebox{1ex}[0pt][0pt]{\sffamily{\@title \ \@date}}} 
\fancyhead[R]{\raisebox{1ex}[0pt][0pt]{\sffamily{\@author}}}
\makeatother

\begin{document}
% sezione (data)
\section{Lezione del 02-10-24}

% stili pagina
\thispagestyle{empty}
\pagestyle{fancy}

% testo
\subsection{Generatori}
I generatori sono i componenti che spostano le cariche attraverso le reti elettriche.
Dividiamo i generatori in due macrocategorie, in base alle loro caratteristiche:
\begin{itemize}
	\item \textbf{Indipendenti:} hanno sempre le stesse caratteristiche, e portano energia all'interno del circuito;
	\item \textbf{Dipendenti:} hanno caratteristiche \textit{pilotate} da altri fattori del circuito, non portano energia in esso e quindi non sono diversi dagli altri dipoli passivi già visti.
\end{itemize}

Inoltre dividiamo entrambe in altre due categorie, in base al tipo di operazione che svolgono:
\begin{itemize}
	\item \textbf{Generatori di tensione:} mantengono i loro capi a differenza di potenziale costante;
	\item \textbf{Generatori di corrente:} mantengono una corrente costante al loro interno.
\end{itemize}

Infine, dividiamo in due ulteriori modalità di operazione:
\begin{itemize}
	\item \textbf{Corrente continua:} mantengono la corrente costante. Si dicono C.C. (Corrente Continua), o D.C. (Direct Current). Il grafico della corrente sarà:
		\begin{center}
\begin{tikzpicture}
    \begin{axis}[
        xlabel={$t$},
        ylabel={$i(t)$},
        domain=-10:10, % set the x range you want
				samples=100,
        grid=major, % add a grid
				ytick={0, 2},
				yticklabels={0, I},
				ymin = -1, ymax = 3,
				width=14cm,
				height=7cm
    ]
    \addplot[
        blue,
        thick
    ] {2}; 
    \end{axis}
\end{tikzpicture}
\end{center}

	\item \textbf{Corrente alternata:} mantengono la corrente in regime sinousidale. Si dicono C.A. (Corrente Alternata), o A.C. (Alternating Current). Il grafico della corrente alternata è stato già visto all'inizio del corso, ha equazione: 
$$
	i(t) = A \sin{\left(\frac{2\pi}{T} t \right)}
$$
con $A$ ampiezza e $T$ periodo, e grafico:

\begin{center}
\begin{tikzpicture}
    \begin{axis}[
        xlabel={$t$},
        ylabel={$i(t)$},
        domain=-10:10, % set the x range you want
				samples=100,
        grid=major, % add a grid
				ytick={-2, 2},
				yticklabels={$-A$, $A$},
				ymin = -3, ymax = 3,
				width=14cm,
				height=7cm
    ]
    \addplot[
        blue,
        thick
    ] {2 * sin(50*x)}; 
    \end{axis}
\end{tikzpicture}
\end{center}
\end{itemize}

Esistono poi altri regimi di applicazione della corrente, che vedremo per casi specifici (impulsi, gradini, ecc...).

Riportiamo intanto ogni combinazione delle prime quattro tipologie nel dettaglio.

\subsubsection{Generatori di tensione}
Un generatore di tensione (o voltaggio) ideale è un componente circuitale che mantiene i suoi capi $A$ e $B$ ad una differenza di potenziale $V_{AB}$ costante, ovvero:
$$ v(t) = E(t) = V $$
dove con $E$ si indica la forza elettromotrice. 
Si indica come:

\begin{center}
\begin{circuitikz}
\draw (0,0) to[ european voltage source ] (2,0); 
\end{circuitikz}
\end{center}

Si nota che a voltaggio nullo, un generatore di tensione equivale a un corto circuito (un filo ideale).

\par\medskip
\noindent
\textbf{\textsf{Correlazione con la corrente}} \\
La tensione erogata da un generatore di tensione è costante, qualsiasi sia la corrente che lo attraversa:
$$ v(i) = \mathrm{const.} $$
Il grafico di correlazione corrente-voltaggio sarà quindi:

\begin{center}
\begin{tikzpicture}
    \begin{axis}[
        xlabel={$i$},
        ylabel={$v$},
        domain=0:4, % set the x range you want
				samples=100,
        grid=major, % add a grid
				ytick={0, 2},
				yticklabels={0, $V$},
				xtick={0},
				ymin = -0.5, ymax = 3,
				xmin = -0.5,
				width=15cm,
				height=7cm
    ]
    \addplot[
        blue,
        thick
    ] {2}; 

    \end{axis}
\end{tikzpicture}
\end{center}

\par\medskip
\noindent
\textbf{\textsf{Correlazione con la potenza}} \\
Tradizionalmente si descrivono i generatori di tensione attraverso riferimenti non associati di corrente e tensione.
Resta il fatto che la potenza:
$$ p(t) = v(t)i(t) = E(t)i(t) $$
quando è erogata dal generatore, è $> 0$.

\par\medskip
\noindent
\textbf{\textsf{Collegamenti in serie}} \\
Per sommare i contributi al voltaggio di più generatori di voltaggio, li disponiamo in serie:

\begin{center}
\begin{circuitikz}
    \draw (0,0) node[left] {$A$} 
        to[european voltage source, l=$V_1$] (2,0) 
        to[european voltage source, l=$V_2$] (4,0) 
        to[short] (5,0)
        node[anchor=west,xshift=0.15cm] {\dots} (6,0) 
        to[short] (7,0)
        to[european voltage source, l=$V_n$] (9,0) node[right] {$B$};

    \draw[decorate,decoration={brace,amplitude=10pt,mirror}] (0,-1) -- (9,-1)
        node[midway,below=10pt]{$n$ generatori};
\end{circuitikz}
\end{center}

Abbiamo che il contributo totale dei generatori equivale a quello di un singolo generatore $E_T$ di voltaggio:
$$ V_T = V_1 + V_2 + ... + V_n $$

\par\medskip
\noindent
\textbf{\textsf{Collegamenti in parallelo}} \\
Non si possono collegare generatori di voltaggio in parallelo, a meno che questi non abbiano lo stesso voltaggio (e quindi risultino in movimento nullo di carica):

\begin{center}
\begin{circuitikz}
    \draw (2,0) 
				to[short] (0,0) node[left] {$A$};
    
    \draw (2,-2) 
				to[short] (0,-2) node[left] {$B$};
		
		\draw (2,0 )to[european voltage source, l=$V_1$] (2,-2);
		\draw (4,0 )to[european voltage source, l=$V_2$] (4,-2);
		
        
    % Draw horizontal lines at the top and bottom
    \draw (0,0) -- (4,0);
    \draw (0,-2) -- (4,-2);
\end{circuitikz}
\end{center}

Dove si ha, dall'applicazione della seconda legge di Kirchoff:
$$
V_1 - V_2 = 0 \Rightarrow V_1 = V_2 
$$
che sarebbe altrimenti violata.

Nella realtà, se si provasse a collegare due generatori di tensione di voltaggio diverso in parallelo, questi proverebbero a imporre la loro differenza di potenziale sui due rami del circuito, creando forti correnti, e probabilmente causando danni termici ad esso o a loro stessi.


\subsubsection{Generatori di corrente}
Un generatore di corrente ideale è un componente circuitale che mantiene attraverso di sé una corrente costante, ovvero:
$$ i(t) = I $$
Si indica come:

\begin{center}
\begin{circuitikz}
\draw (0,0) to[ european current source ] (2,0); 
\end{circuitikz}
\end{center}

Si nota che a corrente nulla, un generatore di corrente equivale a un circuito aperto.

\par\medskip
\noindent
\textbf{\textsf{Correlazione con il voltaggio}} \\
Un generatore di corrente mantiene la stessa corrente qualsiasi sia il voltaggio.
$$ i(v) = \mathrm{const.} $$
Il grafico di correlazione corrente-voltaggio sarà quindi:

\begin{center}
\begin{tikzpicture}
    \begin{axis}[
        xlabel={$i$},
        ylabel={$v$},
        domain=0:4, % set the x range you want
				samples=100,
        grid=major, % add a grid
				ytick={0},
				yticklabels={0},
				xtick={0, 1.5},
				xticklabels={0, $I$},
				ymin = -0.5, ymax = 4,
				xmin = -0.25, xmax = 1.75,
				width=15cm,
				height=7cm
    ]
    \addplot[
        red,
        thick,
        domain=-0.5:3 % set the y range you want
    ] coordinates {(1.5,0) (1.5,3)};


    \end{axis}
\end{tikzpicture}
\end{center}

\par\medskip
\noindent
\textbf{\textsf{Correlazione con la potenza}} \\
Come per i generatori di tensione, si descrivono i generatori di corrente attraverso riferimenti non associati di corrente e tensione.
Resta comunque il fatto che la potenza:
$$ p(t) = v(t)i(t) = v(t)I(t) $$
quando è erogata dal generatore, è $> 0$.

\par\medskip
\noindent
\textbf{\textsf{Collegamenti in serie}} \\
Non si possono collegare generatori di corrente in serie, a meno che questi non abbiano la stessa carica (e quindi risultino in movimento uniforme di carica):

\begin{center}
\begin{circuitikz}
    \draw (0,0) node[left] {$A$} 
        to[european current source, l=$I_1$] (2,0) 
        to[european current source, l=$I_2$] (4,0) node[right] {$B$};
\end{circuitikz}
\end{center}

Dove si ha, dall'applicazione della prima legge di Kirchoff:
$$
I_1 - I_2 = 0 \Rightarrow I_1 = I_2 
$$
che sarebbe altrimenti violata.

Come prima, questa situazione non è effettivamente modellizzabile nella realtà usando il modello studiato.
In verità il generatore di corrente in sé per sé è più uno strumento teorico che serve a modelizzare fenomeni diversi (transistor, amplificatori, ecc...).

\par\medskip
\noindent
\textbf{\textsf{Collegamenti in parallelo}} \\
Per sommare i contributi alla corrente di più generatori di corrente, li disponiamo in parallelo:

\begin{center}
\begin{circuitikz}
    \draw (2,0) 
				to[short] (0,0) node[left] {$A$};
    
    \draw (2,-2) 
				to[short] (0,-2) node[left] {$B$};
		
		\draw (2,0 )to[european current source, l=$I_1$] (2,-2);
		\draw (4,0 )to[european current source, l=$I_2$] (4,-2);
		
		
    \draw (4,0) node[anchor=west,xshift=0.7cm] {\dots} (6,0);
    \draw (4,-2) node[anchor=west,xshift=0.7cm] {\dots} (6,-2); 
		
		\draw (8,0 )to[european current source, l=$I_n$] (8,-2);
        
    % Draw horizontal lines at the top and bottom
    \draw (0,0) -- (4,0);
    \draw (0,-2) -- (4,-2);

    \draw (6,0) -- (8,0);
    \draw (6,-2) -- (8,-2);

    % Bracket below the resistors
    \draw[decorate,decoration={brace,amplitude=10pt,mirror}] (0,-3) -- (8,-3)
        node[midway,below=10pt]{$n$ generatori};
\end{circuitikz}
\end{center}

Abbiamo che il contributo totale dei generatori equivale a quello di un singolo generatore $E_T$ di corrente:
$$ I_T = I_1 + I_2 + ... + I_n $$

\subsubsection{Resistenza interna}
Possiamo combinare i componenti visti finora per creare modelli più realistici.
Innanzitutto, è improbabile che un generatore reali applichi resistenza nulla alle cariche che vi scorrono dentro.
Aggiungiamo quindi una resistenza (solitamente piccola per i generatori di tensione ed elevata per i generatori di corrente) al generatore, che chiameremo \textbf{resistenza interna}.
Questa resistenza rappresenterà la potenza che viene dissipata per effetto Joule.

La resistenza si disporrà come segue per i diversi tipi di generatore:
\begin{itemize}
	\item \textbf{Generatore di tensione:} resistenza in serie;
\begin{center}
\begin{circuitikz}
    \draw (0,0) node[left] {$A$} 
        to[european voltage source, l=$V$] (2,0) 
				to[R, l=$R$] (4,0) node[right] {$B$};
\end{circuitikz}
\end{center}
	\item \textbf{Generatore di corrente:} resistenza in parallelo.
\begin{center}
\begin{circuitikz}
    \draw (2,0) 
				to[short] (0,0) node[left] {$A$};
    
    \draw (2,-2) 
				to[short] (0,-2) node[left] {$B$};
		
		\draw (2,0 )to[european current source, l=$I$] (2,-2);
		\draw (4,0 )to[R, l=$R$] (4,-2);
		
        
    % Draw horizontal lines at the top and bottom
    \draw (0,0) -- (4,0);
    \draw (0,-2) -- (4,-2);
\end{circuitikz}
\end{center}
\end{itemize}

Notiamo che i casi visti prima come impossibili, di generatori di tensione in parallelo e di generatori di corrente in serie, sono rappresentabili quando si rilascia l'ipotesi che i generatori siano ideali e si introducono resistenze interne.

\subsubsection{Generatori dipendenti}
I generatori dipendenti, detti anche controllati o pilotati, sono particolari tipi di generatore il cui voltaggio (o corrente) dipende dal valore del voltaggio (o corrente) di un'altro punto del circuito, scalato di un qualche coefficiente. 
Si indicano come i generatori indipendenti ma all'interno di un rombo invece che di un cerchio.

Abbiamo quindi 4 tipi fondamentali di generatori dipendenti:
\begin{itemize}
	\item \textbf{Generatori di tensione,} si indicano come:
\begin{center}
\begin{circuitikz}
    \draw (0,0) 
        to[european controlled voltage source] (2,0);
\end{circuitikz}
\end{center}
		\begin{itemize}
			\item \textbf{Generatore di tensione pilotato in tensione:} comandato dalla funzione:
				$$
				v(t) = \alpha \cdot v(t)
				$$
				su un punto arbitrario dove si calcola $i(t)$.
			\item \textbf{Generatore di tensione pilotato in corrente:} comandato dalla funzione:
				$$
				v(t) = \alpha \cdot i(t)
				$$
				su un punto arbitrario dove si calcola $v(t)$.
		\end{itemize}
	\item \textbf{Generatori di corrente,} si indicano come:
\begin{center}
\begin{circuitikz}
    \draw (0,0) 
        to[european controlled current source] (2,0);
\end{circuitikz}
\end{center}
		\begin{itemize}
			\item \textbf{Generatore di corrente pilotato in tensione:} comandato dalla funzione:
				$$
				i(t) = \alpha \cdot v(t)
				$$
				su un punto arbitrario dove si calcola $v(t)$.
			\item \textbf{Generatore di corrente pilotato in corrente:} comandato dalla funzione:
				$$
				i(t) = \alpha \cdot i(t)
				$$
				su un punto arbitrario dove si calcola $i(t)$.
		\end{itemize}
\end{itemize}

\par\smallskip 

Bisogna notare che, come già riportato, un generatore dipendente non è diverso da un dipolo passivo in termini di potenza: non porta nessuna energia esterna all'interno del circuito.
Si può anzi dire che è necessario avere almeno un generatore indipendente per avere spostamento di carica all'interno del circuito.

\subsection{Partitore di tensione}
Analizziamo il seguente circuito:

\begin{center}
\begin{circuitikz} 
	\draw
  (0,2) to[V, v=$e(t)$] (0,12) % Voltage source
  -- (2,12) 
  to[R, l=$R_1$] (2,10) 
  to[R, l=$R_2$] (2,8);

	\draw
	(2,7) to[R, l=$R_J$] (2,5);

	\draw
	(2,4) to[R, l=$R_n$] (2,2)
	-- (0, 2);

\draw (2,7.5) node[xshift=0.1] {\dots} (2,5);
	\draw (2,4.5) node[xshift=0.1] {\dots} (2,4); 
\end{circuitikz}
\end{center}

Reti di questo tipo prendono il nome di \textbf{partitori di tensione}, e hanno lo scopo di partizionare una certa differenza di potenziale in diverse frazioni proprie.

Poniamo di voler calcolare la caduta di potenziale su una particolare resistenza, diciamo la $R_J$. Avremo allora, dalla seconda legge di Kirchoff:
$$
-e(t) + R_1(t) i(t)+ R_2(t) i(t) + ... + R_J(t) i(t) + ... + R_n(t) i(t) = 0
$$
che raccogliendo la corrente comune diventa:
$$
e(t) = (R_1 + R_2 + ... + R_J + R_n) i(t) = i(t) \sum_{i=1}^n R_i
$$
somma delle resistenze per corrente.
A questo punto possiamo applicare la legge di Ohm per ottenere la caduta di potenziale:
$$
V_J(t) = R_J i(t) = e(t)\frac{R_j}{\sum_{i=1}^n R_i}
$$
cioè il rapporto fra la resistenza interessata e la resistenza complessiva del circuito, moltiplicata per la tensione.

\subsection{Partitore di corrente}
Analizziamo quindi il seguente circuito:

\begin{center}
\begin{circuitikz}
    \draw (2,-2) 
				-- (0,-2) 
				to[V, v=$e(t)$] (0, 0); % Voltage source
		
		\draw (2,0 )to[R, l=$R_1$] (2,-2);
		\draw (4,0 )to[R, l=$R_2$] (4,-2);
		
		
    \draw (4,0) node[anchor=west,xshift=0.7cm] {\dots} (6,0);
    \draw (4,-2) node[anchor=west,xshift=0.7cm] {\dots} (6,-2); 
		
		\draw (8,0 )to[R, l=$R_J$] (8,-2);
       
    \draw (8,0) node[anchor=west,xshift=0.7cm] {\dots} (10,0);
    \draw (8,-2) node[anchor=west,xshift=0.7cm] {\dots} (10,-2); 
		
		\draw (12,0 )to[R, l=$R_n$] (12,-2);

    % Draw horizontal lines at the top and bottom
    \draw (0,0) -- (4,0);
    \draw (0,-2) -- (4,-2);

    \draw (6,0) -- (8,0);
    \draw (6,-2) -- (8,-2);

    \draw (10,0) -- (12,0);
    \draw (10,-2) -- (12,-2);

\end{circuitikz}
\end{center}

Reti di questo tipo hanno uno scopo simile a quello della rete vista prima, solo riguardo alla corrente: prendono infatti il nome di \textbf{partitori di corrente}.

Poniamo di voler calcolare la corrente su una singola resistenza. 
Potremo dire che la corrente complessiva è, dalla prima legge di Kirchoff:

$$
I_T(t) = I_1(t) + I_2(t) + ... + I_J(t) + ... + I_n(t)
$$

Un'altro modo di ottenere queste correnti è dalla legge di Ohm, usando le conduttanze invece delle resistenze:
$$ 
I = \frac{V}{R} \Rightarrow I = GR, \quad I(t) = v(t) \sum_{i=1}^n G_i
$$
A questo punto, possiamo dire che la corrente nella J-esima resistenza vale:
$$
I_J(t) = v(t) G_n = I(t) \frac{G_J}{\sum_{i=1}^n} 
$$
cioè il rapporto fra la conduttanza (della resistenza) interessata e la conduttanza complessiva del circuito, moltiplicata per la corrente.

\end{document}
